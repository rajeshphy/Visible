\documentclass[a4paper]{article}

\usepackage{amssymb}
\usepackage{siunitx}
\usepackage{amsmath}
\usepackage{bm}
\usepackage{graphicx}

\usepackage{multicol}
\usepackage{savetrees}
\usepackage{hyperref}
\usepackage{etoolbox}

% Compile in SI units
% To compile in CGS units use: toggletrue{cgs}
\newtoggle{cgs}
\togglefalse{cgs}

% Remove table of contents' name
\makeatletter
\renewcommand\tableofcontents{%
  \@starttoc{toc}%
}
\makeatother

% -----------Symbols-----------
\newcommand{\Lagr}{\mathcal{L}}	       % Lagrangian
\newcommand{\Hami}{\mathcal{H}}	       % Hamiltonian
\newcommand{\emf}{\mathcal{E}}	       % EMF
\newcommand{\Imp}{\mathcal{I}}	       % Impulse
\newcommand{\dr}{
  \ensuremath{\text{r}}}               % Displacement r
\newcommand{\dvr}{
  \ensuremath{\textbf{r}}}             % Displacement vector r
\newcommand{\dvrhat}{
  \ensuremath{\ve{\hat{\dvr}}}}	       % Displacement unit vector r

% -----------Functions-----------
\newcommand{\ve}[1]{
  \ensuremath{\bm{#1}}}	               % Vector
\newcommand{\uve}[1]{
  \ensuremath{\bm{\hat{#1}}}}          % Unit Vector
\newcommand{\tensor}[1]{
  \ensuremath{\text{\bf{#1}}}}         % Tensor
\newcommand{\fve}[1]{
  \ensuremath{\vec{#1}}}               % Four-vector
\newcommand{\cc}[1]{
  \ensuremath{#1^{\ast}}}              % Complex Conjugate
\newcommand{\pd}[2]{
  \ensuremath{
    \frac{\partial #1}{\partial #2} }} % Partial Derivative
\newcommand{\operator}[1]{
  \ensuremath{\hat{\text{#1}}}}        % Operator
\newcommand{\ave}[1]{
  \ensuremath{\langle #1 \rangle}}     % Average Value
\newcommand{\bra}[1]{
  \ensuremath{\langle #1 |}}           % Bra of Bra-Ket
\newcommand{\ket}[1]{
  \ensuremath{| #1 \rangle }}          % Ket of Bra-Ket
\newcommand{\braket}[2]{
  \ensuremath{\langle #1|#2 \rangle }} % Bra-Ket


\title{Physics Notes}
\author{Rajesh Kumar}

\begin{document}

\maketitle
% \begin{center}
%   \includegraphics[scale=0.7, natwidth=88 , natheight=31]{deed.png}
% \end{center}

\begin{multicols*}{2}
  \tableofcontents
\end{multicols*}
%\pagebreak
\newpage
\begin{multicols*}{2}
\section{Measurement}
\subsection{Instrument Uncertainty}
All instruments have uncertainties:
\begin{enumerate}
\item Analogue Instruments: Half the smallest measurement unit
\item Digital Instruments: The smallest significant figure
\item Human reaction time: $\pm 0.10$s
\end{enumerate}
\subsection{Significant Figures}
\begin{enumerate}
\item Adding or subtracting: Follow term with least {\em decimal place}
\item Multiplying or Dividing: Follow term with least {\em significant figure}
\end{enumerate}
\subsection{Propagation of error}
For any $f(a, \cdots)$ the general formula for $\Delta f$ is:
\begin{align*}
  \Delta f = \sqrt{\left( \pd{f}{a} \Delta a \right)^2 + \cdots}
\end{align*}
Some specific examples:
\begin{enumerate}
\item $f=a\pm b$
  \begin{align*}
    \Delta f = \sqrt{(\Delta a)^2+(\Delta b)^2}
  \end{align*}
\item $f=ab$ or $f=\frac{a}{b}$
  \begin{align*}
    \frac{\Delta f}{f} = \sqrt{\left(\frac{\Delta a}{a}
      \right)^2+\left(\frac{\Delta b}{b} \right)^2}
  \end{align*}
\end{enumerate}
\section{Mechanics}
\subsection{Statics}
When all objects are motionless (or have constant velocity),
\begin{align*}
  \sum{\ve{F}_{net}}&=0 \\
  \sum{\ve{\tau}_{net}}&=0
\end{align*}
Four basic forces to consider:
\begin{description}
\item[Tension] Pulling force felt by a rope, string, etc. Every piece of rope
  feels a pulling force in both directions.
\item[Friction] Parallel to surface of contact, can be static or kinetic.
\item[Normal] Perpendicular to surface of contact, prevents object from falling
  through surface.
\item[Gravity] Force acting between two objects with mass. Always acts downwards
  for objects on surface of earth.
\end{description}
\subsection{Kinematics}
\begin{align*}
  \ve{v} &= \lim_{\Delta t \rightarrow 0} \frac{\Delta \ve{x}}{\Delta t} = \frac{d\ve{x}}{dt} = 	\ve{\dot{x}} \\
  \ve{a} &= \frac{d\ve{v}}{dt} = \frac{d^2\ve{x}}{dt^2} = \ve{\dot{x}} =
  \ve{\ddot{x}}
\end{align*}
\subsubsection{Polar Coordinates}
Differentiation of unit vectors:
\begin{align*}
  \dot{\uve{r}} &= \dot{\theta} \uve{\theta}\\
  \dot{\uve{\theta}} &= -\dot{\theta} \uve{r}
\end{align*}
Velocity and acceleration in polar form:
\begin{align*}
  \ve{r} &= r\uve{r} \\
  \ve{v} &= \dot{\ve{r}} = \dot{r} \uve{r} + r \dot{\theta} \uve{\theta} \\
  \ve{a} &= \dot{\ve{v}} = (\ddot{r} - \dot{\theta}^2 r) \uve{r} + (r
  \ddot{\theta} + 2 \dot{r} \dot{\theta})\uve{\theta}
\end{align*}
\subsection{Dynamics}
\begin{align*}
  \ve{F} &= m \ve{\ddot{x}} \\
  \ve{F}_{action} &= - \ve{F}_{reaction}
\end{align*}
Free body diagram techniques:
\begin{enumerate}
\item $\Sigma \ve{F}_{net} = 0$ for massless pulleys
\item Conservation of string
\end{enumerate}
Solving differential equations in 1-dimension:
\begin{enumerate}
\item $F=f(t)$
  \begin{align*}
    m \int_{v_0}^{v(t)} dv' &= \int_{t_0}^{t} f(t') dt' \\
    m \int_{x_0}^{x(t)} dx' &= \int_{t_0}^{t} v(t') dt'
  \end{align*}
\item $F=f(x)$
  \begin{align*}
    a= \frac{dv}{dt} = \frac{dv}{dt} \frac{dx}{dx} &= v \frac{dv}{dx} \\
    m \int_{v_0}^{v(x)} v' dv' &= \int_{x_0}^{x} f(x') dx'
  \end{align*}
\item $F=f(v)$
  \begin{align*}
    m \int_{v_0}^{v(t)} \frac{dv'}{f(v')} = \int_{t_0}^{t}dt'
  \end{align*}
\end{enumerate}
\subsubsection{Friction}
Kinetic and static friction:
\begin{align*}
  \ve{f_k}&=\mu_k\ve{N} \\
  \ve{f_s}&\leq\mu_s\ve{N}
\end{align*}
Static friction does no work.
\subsubsection{Constraining Forces}
For any rigid body, there are 6 degrees of freedom ($DF$). There can be
constraining forces ($C$) acting on the body.
\begin{itemize}
\item Statics: $C+DF=6$
\item Dynamics $C+DF \geq 6$
\end{itemize}
There are 3 assumptions made for a body moving without any constraint:
\begin{enumerate}
\item $\ve{f}_{ij} \parallel \ve{r}_{ij}$
\item $\ve{r}_{ij}$ is constant for any 2 points in a rigid body
\item $\ve{f}_{12} + \ve{f}_{21} = 0$
\end{enumerate}
\subsubsection{Fictitious Forces}
For any vector $\ve{A}$ in a moving frame, we calculate its time derivative in a
frame rotating at $\omega$ respect to the stationary frame:
\begin{align*}
  \frac{d\ve{A}}{dt}_{\text{stat}} = \frac{d\ve{A}}{dt}_{\text{mov}} +
  \ve{\omega} \times \ve{A}
\end{align*}
\noindent
Let $\ve{r}$ be the position vector of the object in an accelerated frame and
$\ve{R}$ be the vector to the origin of the accelerated frame, then the possible
forces that acts on $\ve{r}$ in the moving frame are:
\begin{align*}
  \frac{d^2\ve{r}}{dt ^2} = \frac{\ve{F}}{m} &- \frac{d^2\ve{R}}{dt^2}
  - \ve{\omega} \times (\ve{\omega}\times\ve{r})\\
  &- 2\ve{\omega} \times \ve{v} - \frac{d\ve{\omega}}{dt} \ve{r}
\end{align*}
\begin{enumerate}
\item Translational force: $- m\frac{d^2\ve{R}}{dt^2}$
\item Centrifugal force: $-m\ve{\omega} \times (\ve{\omega}\times\ve{r})$
\item Coriolis force: $-2m\ve{\omega} \times \ve{v}$
\item Azimuthal force: $-m \frac{d\ve{\omega}}{dt} \ve{r}$
\end{enumerate}
\subsection{Conservation Laws}
\begin{description}
\item[Energy] $W_{NC} = 0$
\item[Momentum] $\Sigma \ve{F}_{net} = 0$
\item[Angular Momentum] $\Sigma \ve{\tau}_{net} = 0$
\end{description}
\subsection{Energy}
For a force in one dimension:
\begin{align*}
  m \dot{\ve{r}} \frac{d\dot{\ve{r}}}{d\ve{r}} &= \ve{F}(\ve{r}) \\
  \frac{1}{2}m |\dot{\ve{r}}|^2 &= E + \int_{\ve{r_0}}^{\ve{r}} \ve{F}(\ve{r}')
  \cdot d\ve{r}'
\end{align*}
We can then define \emph{potential energy}:
\begin{align*}
  U(\ve{r}) = - \int_{\ve{r_0}}^{\ve{r}} F(\ve{r}') \cdot d\ve{r}'
\end{align*}
Work-Energy theorem:
\begin{align*}
  W_{AB} &= \int_{\ve{r_1}}^{\ve{r_2}} F(\ve{r}') \cdot d\ve{r}' \\
  W_{\text{total}} &= \Delta KE
\end{align*}
Conservative forces are forces that only depend on {\em position}. For
conservative forces:
\begin{align*}
  \oint \ve{F} \cdot d\ve{r} &= 0 \\
  \ve{\nabla} \times \ve{F} &= 0 \\
  \ve{F} &= - \ve{\nabla} U \\
  W_{C} &= -\Delta U
\end{align*}
For non-conservative forces:
\begin{align*}
  W_{NC} = \Delta(K+U) = \Delta E
\end{align*}
Where $E$ is defined as the mechanical energy of the system.
\subsubsection{Virial Theorem}
If we have a collection of particles at positions $\ve{r}_i$, and each of them
experiences a force $\ve{F}_i$, their average kinetic energy is given by:
\begin{align*}
  \ave{ T } &= -\frac{1}{2} \left\ave{ \sum \ve{F}_i \cdot \ve{r}_i \right}
\end{align*}
For one particle:
\begin{align*}
  \ave{ T } &= -\frac{1}{2} \left\ave{ \frac{dU}{dr} \cdot \ve{r} \right}
\end{align*}
\subsubsection{Power}
Power is the rate of work done per unit time:
\begin{align*}
  P=\frac{dW}{dt}
\end{align*}
Mechanical power:
\begin{align*}
  P=\frac{d}{dt}\oint \ve{F} \cdot d\ve{r}&=\frac{d}{dt}\oint \ve{F} \cdot \frac{d\ve{r}}{dt} dt\\
  &=\ve{F} \cdot \ve{v}
\end{align*}
\subsection{Momentum}
Momentum is defined as:
\begin{align*}
  \ve{p} = m \ve{v}
\end{align*}
When there is no net force on the system,
\begin{align*}
  \sum \ve{F}_{net} = 0 &\Rightarrow \frac{d\ve{p}}{dt} = 0 \\
  &\Rightarrow \ve{p} \text{ is conserved}
\end{align*}
Impulse is defined as:
\begin{align*}
  \Imp &= \int_{t_1}^{t_2} \ve{F}(t) dt = \int_{t_1}^{t_2} \frac{d\ve{p}}{dt} dt \\
  \Imp &= \ve{p}(t_2) - \ve{p}(t_1) = \Delta \ve{p}
\end{align*}
For perfectly elastic collisions of two objects in 1-D, relative velocity is
constant.
\begin{align*}
  \ve{v}_1 - \ve{v}_2 = - (\ve{v}'_1 - \ve{v}'_2)
\end{align*}
For other collisions in 1-D, we have the coefficient of restitution $e$:
\begin{align*}
  e = -\frac{\ve{v}'_2 - \ve{v}'_1}{\ve{v}_2 - \ve{v}_1} \qquad 0 \leq e \leq 1
\end{align*}
\subsection{Lagrangian Mechanics}
The Lagrangian method is based on the \emph{principle of stationary action}.
\begin{align*}
  \Lagr(\dot{x},x,t) = T - V \\
  \frac{d}{dt}\left( \pd{\Lagr}{\dot{x}} \right) - \pd{\Lagr}{x} = 0
\end{align*}
\subsubsection{Multiple Coordinates}
If we have a Lagrangian in $n$ coordinates $\Lagr(t, q_1, \dot{q_1}, \cdots,
q_n, \dot{q_n})$, we simply get $n$ Euler-Lagrange equations:
\begin{align*}
  \frac{d}{dt}\left( \pd{\Lagr}{\dot{q}_i} \right) = \pd{\Lagr}{q_i}
\end{align*}
\subsubsection{Forces of Constraint}
If we have an equation of constraint $f(\ve{x}) = 0$, we can use Lagrange
multipliers to get the equations of motion, and along with the constraint
equations solve for $\lambda$:
\begin{align*}
  \pd{\Lagr}{x_i} + \lambda \pd{f}{x_i} = \frac{d}{dt} \left(
    \pd{\Lagr}{\dot{x}_i} \right)
\end{align*}
The the forces of constraint are:
\begin{align*}
  F_i^c = \lambda \pd{f}{x_i}
\end{align*}
\subsubsection{Conservation of Energy}
If we take the total time derivative of the Lagrangian, we get:
\begin{align*}
  \frac{d\Lagr}{dt} = \pd{\Lagr}{t} + \ddot{q}_i\pd{\Lagr}{\dot{q}_i} +
  \dot{q}_i \frac{d}{dt} \left( \pd{\Lagr}{\dot{q}_i} \right)
\end{align*}
If the Lagrangian is explicitly independent of time, we have the following conserved quantity, which is the energy of the system:
\begin{align*}
  \frac{d}{dt} \left[ \dot{q}_i \pd{\Lagr}{\dot{q}_i} - \Lagr \right] = 0
\end{align*}
\subsubsection{Noether's Theorem}
A ``symmetry'' is a change of coordinates that does not result in a first order
change in the Lagrangian. For each symmetry, there is a conserved quantity. If
the Lagrangian is invariant in first order under the change of coordinates:
\begin{align*}
  q_i \rightarrow q_i + \epsilon K_i (q)
\end{align*}
The following quantity is conserved:
\begin{align*}
  \frac{d}{dt} \left[ K_i(q) \pd{\Lagr}{\dot{q_i}} \right]
\end{align*}
\subsection{Hamiltonian Mechanics}
The Hamiltonian $\Hami(\ve{p},\ve{q},t)$ for multiple coordinates is defined as:
\begin{align*}
  \Hami &= p_i \dot{q}_i - \Lagr(\ve{q}, \dot{\ve{q}}, t) \\
  p_i &= \pd{\Lagr}{\dot{q}_i}
\end{align*}
The following equations of motion can then be obtained:
\begin{align*}
  \pd{\Hami}{q_i} &= -\dot{p}_i \\
  \pd{\Hami}{p_i} &= \dot{q}_i \\
  \pd{\Hami}{t} &= - \pd{\Lagr}{t}
\end{align*}
\subsubsection{Liouville's Theorem}
The Hamiltonian formulation gives two first order ordinary differential
equations, which can always be uniquely solved when given initial conditions
$(\ve{p}_0, \ve{q}_0)$. Thus no two phase space orbits with different initial
conditions cross, and consequently any volume in phase space is constant under
time evolution.
\subsubsection{Poisson Brackets}
The Poisson bracket binary operation is defined as:
\begin{align*}
  \{ f(p, q, t), g(p, q, t) \} = \pd{f}{q}\pd{g}{p} - \pd{f}{p}\pd{g}{q}
\end{align*}
If $p$ and $q$ are the solutions to Hamiltonian's equations:
\begin{align*}
  \dot{q} &= \{q, \Hami \} \\
  \dot{p} &= \{p, \Hami \} \\
  \frac{d}{dt} f(p, q, t) &= \{f, \Hami\} + \pd{f}{t}
\end{align*}
\subsection{Central Forces}
For any two objects subject to a central force,
\begin{align*}
  F(r) &= \mu \ddot{r} - \mu r \dot{\theta}^2\\
  L &= \mu r^2\dot{\theta}
\end{align*}
Where $\mu = (m_1 m_2)/(m_1 + m_2)$ is their reduced mass. Because angular
momentum $L$ is constant, we can look at central forces systems in 1-dimension.
\begin{align*}
  V_{\text{eff}}(r) &= \frac{L^2}{2 \mu r^2} + V(r) \\
  E &= V_{\text{eff}} + \frac{1}{2} \mu \dot{r}^2
\end{align*}
If we let $q(\theta) = \frac{1}{r}$, we get the following equation in polar
coordinates:
\begin{align*}
  q''(\theta) + q(\theta) + \frac{\mu}{L^2 q^2} F(r) = 0
\end{align*}
\subsubsection{Gravity}
For any two point masses of $m_1$ and $m_2$ in empty space, the gravitational
force between them is:
\begin{align*}
  \ve{F}=\frac{Gm_1m_2}{|\ve{r}|^2}\uve{r}
\end{align*}
Where $\ve{r}$ is the position vector of one mass respect to the other, and $G$
is the gravitational constant.
\begin{align*}
  F=mg
\end{align*}
\noindent For a mass $m$ at the Earth's surface, where $g=9.81m/s^2$ pointing
downwards.
\subsection{Uniform Circular Motion}
For a point mass moving in uniform circular motion, we define:
\begin{align*}
  \omega=\frac{v}{r}
\end{align*}
The centripetal acceleration $a$ and the force required to keep the object in
its circular path:
\begin{align*}
  a&=\frac{v^2}{r}=\omega^2r\\
  F&=\frac{mv^2}{r}=m\omega^2r
\end{align*}
\subsection{\texorpdfstring{Rotational Dynamics (Constant $\uve{L}$)}{Rotational
    Dynamics (Constant Direction of L)}} %Prevents \hyperref error
\subsubsection{Angular Momentum}
The angular momentum of a point mass is defined as:
\begin{align*}
  \ve{L}=\ve{r}\times\ve{p}
\end{align*}
For a flat object lying on a 2-D plane rotating with angular speed $\omega$:
\begin{align*}
  \ve{L}=\int\ve{r}\times\ve{p}=\int r^2\omega \uve{z}dm
\end{align*}
If we define the {\em moment of intertia} about the $z$-axis to be $I_z=\int
(x^2+y^2)dm$, we have:
\begin{align*}
  L_z&=I_z\omega\\
  T&=\int\frac{1}{2}m\ve{v}^2=\int\frac{r^2\omega^2}{2}dm\\
  &=\frac{1}{2}I_z\omega^2
\end{align*}
For the $z$-component of $\ve{L}$ and kinetic energy $T$.
\subsubsection{General Motion}
For an object with a moving center of mass, and rotating at $\omega$ about it,
\begin{align*}
  \ve{L}&=\ve{r_\text{CM}}\times\ve{p_\text{CM}}+I_\text{CM}\omega \uve{z}\\
  T&=\frac{1}{2}mv_\text{CM}^2+\frac{1}{2}I_\text{CM}\omega^2
\end{align*}
\subsubsection{Torque}
Torque is defined as:
\begin{align*}
  \ve{\tau}=\ve{r}\times\ve{F}
\end{align*}
Using an origin satisfying any of the following conditions to calculate $\ve{L}$,
\begin{enumerate}
\item The origin is the center of mass
\item The origin is not accelerating
\item $(\ve{R}-\ve{r_0})$ is parallel to $\ve{r_0}$, the position of the origin
  in a fixed coordinate system
\end{enumerate}
\begin{align*}
  \frac{d\ve{L}}{dt}=\sum \ve{\tau_\text{ext}}
\end{align*}
When there is no external torque, we have the conservation of angular momentum.
\begin{align*}
  \ve{\tau_\text{ext}}=I\alpha
\end{align*}
Where $\alpha=\frac{d\omega}{dt}$ is the angular acceleration.
\subsubsection{Angular Impulse}
Angular impulse is defined as:
\begin{align*}
  \Imp_\theta=\int_{t_1}^{t_2}\ve{\tau}(t)dt=\Delta\ve{L}
\end{align*}
If we have a force $\ve{F}(t)$ applied at a constant distance $R$ from the
origin,
\begin{align*}
  \ve{\tau}(t)&=\ve{R}\times\ve{F}(t) \\
  \Imp_\theta&=\ve{R}\times\Imp \\
  \Delta\ve{L}&=\ve{R}\times(\Delta\ve{p})
\end{align*}
\subsubsection{Parallel-axis Theorem}
Let an object of mass $M$ rotate about its center of mass with the same
frequency $\omega$ as the center of mass rotates about the origin (with radius
$R$):
\begin{align*}
  L_z=(MR^2+I_\text{CM})\omega
\end{align*}
Thus if the moment of inertia of an object is $I_0$ about a particular axis, its
moment of inertia about a parallel axis separated by $R$ is:
\begin{align*}
  I=MR^2+I_0
\end{align*}
\subsubsection{Perpendicular-axis Theorem}
For flat 2-D objects in the $x$-$y$ plane, and orthogonal axes $x$, $y$ and $z$:
\begin{align*}
  I_z=I_x+I_y
\end{align*}
\subsubsection{Moments of Inertia}
Center of mass for an object of mass $M$:
\begin{align*}
  \ve{R_\text{CM}}=\frac{\int\ve{r}dm}{M}
\end{align*}
Common moments of inertia (taken about center of mass unless stated):
\begin{enumerate}
  \setlength{\itemsep}{2mm}
\item Point mass at $r$ from axis: $mr^2$
\item Rod of length $L$ about center: $\frac{1}{13}mL^2$
\item Rod of length $L$ about one end: $\frac{1}{3}mL^2$
\item Solid disk of radius $r$ perpendicular to axis: $\frac{1}{2}mr^2$
\item Hollow sphere with radius $r$: $\frac{2}{3}mr^2$
\item Solid sphere with radius $r$: $\frac{2}{5}mr^2$
\end{enumerate}
\subsection{General Rotational Motion}
For any body moving in space, its motion can be written as a sum of its
translational motion and a rotation about an axis at a particular time.
\subsubsection{Angular Velocity}
The angular velocity vector $\ve{\omega}$ points along the axis of rotation,
with a magnitude equal to its angular speed. Its direction is determined by
convention of the right hand rule. For an object rotating at $\ve{\omega}$, the
time derivative of any vector $\ve{r}$ fixed in the body frame is:
\begin{align*}
  \ve{v} = \frac{d\ve{r}}{dt} = \ve{\omega} \times \ve{r}
\end{align*}
Angular velocities add like vectors. Let $S_1$, $S_2$ and $S_3$ be coordinate
systems. If $S_1$ rotates with $\ve{\omega}_{1, 2}$ with respect to $S_2$, and
$S_2$ rotates with $\ve{\omega}_{2, 3}$ with respect to $S_3$, then $S_1$
rotates instantaneously with respect to $S_3$ at:
\begin{align*}
  \ve{\omega}_{1, 3} = \ve{\omega}_{1, 2} + \ve{\omega}_{2, 3}
\end{align*}
\subsubsection{Angular Momentum}
\begin{align*}
  \ve{L} &= \int \ve{r} \times (\ve{\omega} \times \ve{r}) dm \\
  &= \tensor{I} \ve{\omega}
\end{align*}
$\tensor{I}$ is the moment of inertia tensor:
\vspace*{-0.7em}
\begin{center}
  \resizebox{\hsize}{!}{%
    $\begin{pmatrix}
      \int(y^2 +z^2) & -\int xy        & -\int zx        \\
      -\int xy       & \int(z^2 + x^2) & -\int yz        \\
      -\int zx       & -\int yz        & \int(x^2 + y^2) \\
    \end{pmatrix}$%
  }
\end{center}
The kinetic energy of the object is given by:
\begin{align*}
  T &= \int\frac{1}{2} ||\ve{\omega} \times \ve{r}||^2 dm \\
  &= \frac{1}{2} \ve{\omega} \cdot \tensor{I}\ve{\omega} = \frac{1}{2} \ve{\omega} \cdot \ve{L}
\end{align*}
To find the angular momentum for an object of mass $M$ in general motion, let
the position of its center of mass be $\ve{R}$, its velocity be $\ve{V}$. Then:
\begin{align*}
  \ve{L} = M(\ve{R} \times \ve{V}) + \ve{L}_{\text{CM}}
\end{align*}
The kinetic energy of the object is:
\begin{align*}
  T = \frac{1}{2}MV^2 + \frac{1}{2}\ve{\omega}' \ve{L}_{\text{CM}}
\end{align*}
Where $\ve{\omega}'$ and $\ve{L}_{\text{CM}}$ are measured about the center of
mass along axes parallel to the fixed-frame axes.
\subsubsection{Principle Axes}
A principle axis is an axis of rotation $\uve{\omega}$ such that
$\tensor{I}\uve{\omega} = I\uve{\omega}$. An object can rotate about a principle
axis at constant angular velocity with no external torque. An orthonormal set of
principle axis exists for every object.
\subsubsection{Euler's Equations}
When an object is instantaneously rotating about an axis $\ve{\omega}$, we can
relate the rate of change of angular momentum in the frame of the principle axes
and the lab frame by:
\begin{align*}
  \frac{d\ve{L}}{dt}_\text{lab} = \frac{d\ve{L}}{dt}_\text{body} + \ve{\omega}
  \times \ve{L}
\end{align*}
This gives us Euler's equations, where $\omega_i$ and $\tau_i$ are components of
$\ve{\omega}$ and torque projected onto the principle axes respectively:
\begin{align*}
  \tau_1 &= I_1\dot{\omega_1} - (I_2 - I_3)\omega_2\omega_3 \\
  \tau_2 &= I_1\dot{\omega_2} - (I_3 - I_1)\omega_3\omega_1 \\
  \tau_3 &= I_1\dot{\omega_3} - (I_1 - I_2)\omega_1\omega_2 \\
\end{align*}
\section{Special Relativity}
\subsection{Postulates}
\begin{enumerate}
\item The speed of light has the same value in all inertial frames
\item Physical laws remain the same in all inertial frames
\end{enumerate}
\subsection{Kinematics}
\subsubsection{Lorentz Transform}
\begin{align*}
  x&= \gamma (x' + \beta ct') \\
  y&=y' \\
  z&=z' \\
  ct&= \gamma (\beta x' + ct')
\end{align*}
Where $\gamma = \frac{1}{\sqrt{1-\frac{v^2}{c^2}}}$ and $\beta = \frac{v}{c}$.
\subsubsection{Fundamental Effects}
\begin{description}
  \setlength{\itemsep}{2.5mm}
\item[Length contraction]
  \begin{align*}
    l'=\frac{l}{\gamma}
  \end{align*}
  Where $l$ is the proper length.
\item[Time dilation]
  \begin{align*}
    t'=\gamma t
  \end{align*}
  Where $t$ is the proper time.
\item [Loss of simultaneity]
  \begin{align*}
    \Delta t = \frac{Lv}{c^2}
  \end{align*}
  Two events separated by $L$ and $\Delta t$ in the rest frame will appear
  simultaneous to an observer moving at $v$.
\item[Longitudinal velocity addition]
  \begin{align*}
    v_x'=\frac{u+v}{1+uv/c^2}
  \end{align*}
  Where $u$ is the velocity of an object in the frame traveling at $v$ respect
  to the lab frame, and $v_x'$ is the $x$-velocity of the object viewed by the
  lab frame.
\item[Transverse velocity addition]
  \begin{align*}
    v_y'=\frac{u_y}{\gamma_v ( 1+ u_xv/c^2)}
  \end{align*}
  Where $u_y$ and $u_x$ are velocity components of an object in the frame
  traveling at $v$ respect to the lab frame, and $v_y'$ is the $y$-velocity of
  the object viewed by the lab frame.
\item[Longitudinal Doppler effect]
  \begin{align*}
    f'=f\sqrt{\frac{1+\beta}{1-\beta}}
  \end{align*}
  Where $f'$ is the frequency observed of a moving source emitting at frequency
  $f$ in its rest frame.
\end{description}
\subsubsection{Minkowski Diagrams}
Space-time diagrams with $x$ and $ct$ axes. Some properties are:
\begin{enumerate}
\item Light travels at $\ang{45}$ to horizontal.
\item $x'$ and $ct'$ axes of another moving frame are $\theta$ to the $x$ and
  $ct$ axes respectively, with
  \begin{align*}
    \tan(\theta)=\beta
  \end{align*}
\item Units on axes of the moving and stationary frames are related by:
  \begin{align*}
    \frac{x'}{x}=\frac{ct'}{ct}=\sqrt{\frac{1+\beta^2}{1-\beta^2}}
  \end{align*}
\end{enumerate}
\subsection{Dynamics}
\subsubsection{Momentum}
\begin{align*}
  \ve{p}=\gamma_vm\ve{v}=\frac{m\ve{v}}{\sqrt{1-\frac{v^2}{c^2}}}
\end{align*}
\subsubsection{Energy}
\begin{align*}
  E^2=p^2c^2+m^2c^4
\end{align*}
For massive particles:
\begin{align*}
  E&=\gamma mc^2=\frac{mc^2}{\sqrt{1-\frac{v^2}{c^2}}}
\end{align*}
For massless particles(such as photons):
\begin{align*}
  E=pc=\frac{hc}{\lambda}
\end{align*}
\subsection{4-vectors}
A 4-vector $\fve{A}=(A_0,A_1,A_2,A_3)$ is a quantity that transforms as follows:
\begin{align*}
  A_0'&= \gamma(A_0+\beta A_1)\\
  A_1'&= \gamma(A_1+\beta A_0)\\
  A_2'&= A_2\\
  A_3'&= A_3
\end{align*}
The dot product of two 4-vectors is invariant under Lorentz transformations:
\begin{align*}
  \fve{A}\cdot\fve{B} &= A_1B_1 + A_2B_2 + A_3B_3 - A_0B_0\\
  &= \fve{A'}\cdot\fve{B'}
\end{align*}
\subsubsection{Different 4-vectors}
\begin{description}
  \setlength{\itemsep}{-2mm}
\item[4-position] $(cdt, dx, dy, dz)$
  4-vectors originate from the invariant interval $ds$.
  \begin{align*}
    \fve{ds}^2&=(cdt, dx, dy, dz)^2 \\
    &=dx^2+dy^2+dz^2-c^2dt^2
  \end{align*}
\item[4-velocity] $\gamma_v(c, \ve{v})$
  To obtain other 4-vectors, we can multiply invariant quantities to the
  4-position vector, such as proper time:
  \begin{align*}
    d\tau&=\frac{dt}{\gamma}\\
    \fve{v}&=\frac{ds}{d\tau}\\
    &=\gamma_v\left(c, \frac{dx}{dt},\frac{dy}{dt},\frac{dz}{dt}\right)\\
    &=\gamma_v(c, \ve{v})
  \end{align*}
\item[4-momentum] $\left(\frac{E}{c}, \ve{p}\right)$
  As mass is invariant,
  \begin{align*}
    \fve{p}&=m\fve{v}\\
    &=(\gamma_vm\ve{v},\gamma_vmc)\\
    &=\left(\frac{E}{c},\ve{p}\right)
  \end{align*}
  For photons in x-direction, the 4-momentum vector is:
  \begin{align*}
    \fve{p}=\left(\frac{h}{\lambda},\frac{h}{\lambda},0,0\right)
  \end{align*}
\item[4-wave] $\left(\frac{\omega}{c},\ve{k}\right)$
  For electromagnetic waves,
  \begin{align*}
    k&=\frac{2\pi}{\lambda}=\frac{\omega}{c}\\
    \ve{p}&=\frac{h}{\lambda}=\hbar\ve{k}\\
    E&=hf=\hbar\omega\\
    \fve{p}&=\hbar\left(\frac{\omega}{c},\ve{k}\right)\\
    \fve{k}&=\frac{\fve{p}}{\hbar}
  \end{align*}
\item[4-acceleration] $\gamma_v^4\left(va_x, a_x, \frac{a_y}{\gamma_v^2},
    \frac{a_z}{\gamma_v^2}\right)$
  \begin{align*}
    \fve{a}&=\frac{d\fve{v}}{d\tau}\\
    &=\gamma_v^4\left(va_x, a_x,\frac{a_y}{\gamma_v^2},
      \frac{a_z}{\gamma_v^2}\right)
  \end{align*}
\item[4-force] $\gamma_v\left(\frac{1}{c}\frac{dE}{dt},\ve{f}\right)$
  \begin{align*}
    \fve{F}&=\frac{d\fve{p}}{d\tau}\\
    &=\gamma_v\left(\frac{d}{dt}\frac{E}{c},\ve{f}\right)
  \end{align*}
\end{description}
\section{Electricity and Magnetism}
\subsection{Electrostatics}
\begin{description}
\item[Coulomb's law] The force between a point charge $q$ and test charge $Q$:
  \begin{align*}
    \iftoggle{cgs}{
    \ve{F} =\frac{Qq}{\dr^2}\dvrhat
    }{
    \ve{F} =\frac{1}{4\pi\epsilon_0}\frac{Qq}{\dr^2}\dvrhat
    }
  \end{align*}
  Where $\dvr=\ve{r}-\ve{r'}$ is the displacement vector from $Q$ at $\ve{r}$
  and $q$ at $\ve{r'}$.
\item[Superposition principle] The interaction between any two charges is
  unaffected by any other charges
\end{description}
\subsubsection{Electric Field}
The electric field of a point charge is defined as:
\begin{align*}
  \iftoggle{cgs}{
    \ve{E}=\frac{\ve{F}}{Q}=\frac{q}{\dr^2}\dvrhat
  }{
    \ve{E}=\frac{\ve{F}}{Q}=\frac{1}{4\pi\epsilon_0}\frac{q}{\dr^2}\dvrhat
  }
\end{align*}
For a continuous volume charge distribution $\rho(\ve{r'})$, we can use the
superposition principle to get:
\begin{align*}
  \iftoggle{cgs}{
    \ve{E}(\ve{r}) = \int_\mathcal{V} \frac{\rho(\ve{r'})} {\dr^2} \dvrhat d\tau'
  }{
    \ve{E}(\ve{r}) = \frac{1}{4\pi\epsilon_0} \int_\mathcal{V}
    \frac{\rho(\ve{r'})} {\dr^2} \dvrhat d\tau'
  }
\end{align*}
Taking the divergence of $\ve{E}$, we get Gauss' law:
\begin{align*}
  \iftoggle{cgs}{
  \ve{\nabla}\cdot\ve{E} &= 4 \pi \rho{(\ve{r})}\\
  \oint_\mathcal{S}\ve{E}\cdot d\ve{a} &= 4 \pi Q_\text{enc}
  }{
  \ve{\nabla}\cdot\ve{E}&=\frac{\rho{(\ve{r})}}{\epsilon_0}\\
  \oint_\mathcal{S}\ve{E}\cdot d\ve{a}&=\frac{Q_\text{enc}}{\epsilon_0}
  }
\end{align*}
Taking the curl of $\ve{E}$:
\begin{align*}
  \ve{\nabla}\times\ve{E}&=0\\
  \oint\ve{E}\cdot d\ve{l}&=0
\end{align*}
For any surface charge in an electric field $\ve{E}$, the field felt by an area
element on the surface is:
\begin{align*}
  \ve{E}_\text{felt}=\frac{1}{2}\left(\ve{E}_\text{above}+\ve{E}_\text{below}\right)
\end{align*}
\subsubsection{Electric Potential}
As the line integral of the electrostatic field is path independent, we can
define the potential at a point $\ve{r}$:
\begin{align*}
  V(\ve{r})=-\int_{\mathcal{O}}^{\ve{r}}\ve{E}\cdot d\ve{l}
\end{align*}
Where $\mathcal{O}$ is a standard reference point, usually set to infinity. The
potential of a point charge can then be found, and with the superposition
principle we can find the potential of any charge distribution:
\begin{align*}
  \iftoggle{cgs}{
    V&=\frac{q}{\dr}\\
    V(\ve{r})&=\int_\mathcal{V}\frac{\rho(\ve{r'})}{\dr} d\tau' }{
    V&=\frac{1}{4\pi\epsilon_0}\frac{q}{\dr}\\
    V(\ve{r})&=\frac{1}{4\pi\epsilon_0}\int_\mathcal{V}\frac{\rho(\ve{r'})}{\dr}
    d\tau' }
\end{align*}
Taking the gradient of the potential:
\begin{align*}
  \ve{E}&=- \ve{\nabla} V\\
  \iftoggle{cgs}{
  \nabla^2V&= -4 \pi \rho
  }{
  \nabla^2V&=-\frac{\rho}{\epsilon_0}
  }
\end{align*}
\subsubsection{Work and Energy}
The work needed to bring a charge $Q$ from infinity to a point $\ve{a}$ is:
\begin{align*}
  W&=\int_{\infty}^{\ve{a}} \ve{F}\cdot d\ve{l}\\
  &=-Q\int_{\infty}^{\ve{a}} \ve{E}\cdot d\ve{l}\\
  &=QV(\ve{a})
\end{align*}
The energy in a continuous charge distribution is:
\begin{align*}
  W&=\frac{1}{2}\int \rho V d\tau\\
  \iftoggle{cgs}{
  &=\frac{1}{8 \pi} \int E^2 d\tau
  }{
  &=\frac{\epsilon}{2} \int E^2 d\tau
  }
\end{align*}
Where the integral is taken over all space.
\subsubsection{Conductors}
A perfect conductor has an unlimited supply of free charges.
\begin{enumerate}
\item $\ve{E}=0$ and $\rho = 0$ inside a conductor
\item Any conductor is an equipotential
\item Just outside a conductor, $\ve{E}$ is perpendicular to the surface.
\end{enumerate}
If we charge up two conductors with $+Q$ and $-Q$, the potential between them is
proportional to the charge $Q$ (because the electric field is proportional to
$Q$), and we define the constant of proportionality capacitance:
\begin{align*}
  C=\frac{Q}{V}
\end{align*}
The work done by charging a capacitor is:
\begin{align*}
  W=\int_0^Q \frac{q}{C} dq &= \frac{Q^2}{2C}\\
  &=\frac{1}{2}CV^2
\end{align*}
\subsubsection{Image Charges}
In certain special cases, a charge placed next to a grounded conductor has
equivalents.
\begin{enumerate}
\item A point charge and a conducting sheet: An opposite charge in the mirror
  image position.
\item A point charge and a conducting sphere, or an infinite line charge and
  conducting cylinder: Opposite image charge and charge forms the Apollonius
  sphere/cylinder.
\end{enumerate}
\subsubsection{Uniqueness Theorems}
\begin{description}
\item[First uniqueness theorem] The solution to Laplace's equation
  ($\nabla^2V=0$) in some volume $\mathcal{V}$ is uniquely determined if $V$ is
  specified on the boundary surface $\mathcal{S}$.
\item[Second uniqueness theorem] In a volume $\mathcal{V}$ surrounded by
  conductors and containing a specified charge density $\rho$, the electric
  field is uniquely determined if the total charge on each conductor is given.
\end{description}
\subsection{Magnetostatics}
\subsubsection{Lorentz Force Law}
The force felt by:
\begin{enumerate}
\item A point charge $q$ moving at velocity $\ve{v}$ through a magnetic field
  $\ve{B}$:
  \begin{align*}
    \iftoggle{cgs}{
    \ve{F}=\frac{q}{c}\ve{v}\times\ve{B}
    }{
    \ve{F}=q\ve{v}\times\ve{B}
    }
  \end{align*}
\item A line current $\ve{I}$:
  \begin{align*}
    \iftoggle{cgs}{
    \ve{F}=\frac{I}{c}\int(d\ve{l}\times\ve{B})
    }{
    \ve{F}=I\int(d\ve{l}\times\ve{B})
    }
  \end{align*}
\item A general volume current $\ve{J}$ per unit area perpendicular to flow:
  \begin{align*}
    \iftoggle{cgs}{
    \ve{F}=\frac{1}{c}\int(\ve{J}\times\ve{B})d\tau
    }{
    \ve{F}=\int(\ve{J}\times\ve{B})d\tau
    }
  \end{align*}
\end{enumerate}
\subsubsection{Biot-Savart Law}
The magnetic field created by a steady line current:
\begin{align*}
  \iftoggle{cgs}{
  \ve{B}(\ve{r})= \frac{I}{c}\int\frac{d\ve{l}\times\dvrhat}{\dr^2}
  }{
  \ve{B}(\ve{r})= \frac{\mu_0}{4\pi}I\int\frac{d\ve{l}\times\dvrhat}{\dr^2}
}
\end{align*}
\subsubsection{Magnetic Fields}
Unlike in electrostatics, conductors do not screen magnetic fields. The magnetic field is divergence-free:
\begin{align*}
  \ve{\nabla}\cdot\ve{B}&=0\\
  \oint\ve{B}\cdot d\ve{a}&=0
\end{align*}
Taking the curl of the magnetic field gives Ampere's Law:
\begin{align*}
  \iftoggle{cgs}{
  \ve{\nabla}\times\ve{B}&=\frac{4\pi}{c}\ve{J}\\
  \oint\ve{B}\cdot d\ve{l}&=\frac{4 \pi}{c} I_{\text{enc}}
  }{
  \ve{\nabla}\times\ve{B}&=\mu_0\ve{J}\\
  \oint\ve{B}\cdot d\ve{l}&=\mu_0 I_{\text{enc}}
  }
\end{align*}
The energy stored in an magnetic field is:
\begin{align*}
  \iftoggle{cgs}{
    U = \frac{1}{8\pi} \int B^2 d\tau
  }{
    U = \frac{1}{2\mu_0} \int B^2 d\tau
  }
\end{align*}
\subsubsection{Magnetic Vector Potential}
For any magnetic field $\ve{B}$, we define the vector potential $\ve{A}$ such
that:
\begin{align*}
  \ve{B} = \ve{\nabla} \times \ve{A}
\end{align*}
Taking the curl of the magnetic field and applying Ampere's law, we get:
\begin{align*}
  \iftoggle{cgs}{
    \nabla^2 \ve{A} &= - \frac{4 \pi}{c} \ve{J} \\
    \ve{A} &= \frac{1}{c} \int \frac{\ve{J}}{r} dV
  }{
    \nabla^2 \ve{A} &= - \mu_0 \ve{J} \\
    \ve{A} &= \frac{\mu_0}{4 \pi} \int \frac{\ve{J}}{r} dV
  }
\end{align*}
\subsection{Electrodynamics}
\subsubsection{Electric Currents}
For $N_e$ particles of charge $e$ moving at an average velocity $\ave{ \ve{v}}$,
the current density is:
\begin{align*}
  \ve{J} = -N_e e \ave{ \ve{v} }
\end{align*}
If $\rho$ is the charge density, the current $I$ is given by:
\begin{align*}
  I = - \pd{}{t} \int_V \rho d\tau
\end{align*}
When the total charge is conserved, we have the continuity equation:
\begin{align*}
  \ve{\nabla} \cdot \ve{J} = - \pd{\rho(t)}{t}
\end{align*}
\subsubsection{Electromotive Force}
For an electric field applied in a material:
\begin{align*}
  \ve{J} = \sigma \ve{E}
\end{align*}
Where $\sigma$ is the conductivity constant depending on the material.
This leads to Ohm's law:
\begin{align*}
  V&=IR\\
  R&=\frac{l}{\sigma A}
\end{align*}
The power delivered:
\begin{align*}
  P=VI=I^2R
\end{align*}
The electromotive force (emf) $\emf$ is the line integral of the force per unit
charge driving the current:
\begin{align*}
  \emf &=\oint \ve{E} \cdot d\ve{l}
\end{align*}
$\emf=V$ for an ideal source.
\subsubsection{Faraday's Law}
Faraday's law states that a changing magnetic flux $\Phi$ induces an electric field:
\begin{align*}
  \iftoggle{cgs}{
    \emf =\oint \ve{E} \cdot d\ve{l} &= -\frac{1}{c}\frac{d\Phi}{dt}\\
    \ve{\nabla}\times\ve{E} &= -\frac{1}{c} \frac{d\ve{B}}{dt}
  }{
    \emf =\oint \ve{E} \cdot d\ve{l} &= -\frac{d\Phi}{dt}\\
    \ve{\nabla}\times\ve{E} &= -\frac{d\ve{B}}{dt}
  }
\end{align*}
\subsubsection{Inductance}
If we have two current loops 1 and 2, the flux $\Phi_2$ through loop 2 is
proportional to the current through loop 1:
\begin{align*}
  \iftoggle{cgs}{
    \Phi_2 = cM_{21}I_1
  }{
    \Phi_2 = M_{21}I_1
  }
\end{align*}
Where $M_{21}=M_{12}$ is the mutual inductance between these two loops. We can
also define an self inductance $L$, for a single loop:
\begin{align*}
  \iftoggle{cgs}{
    \Phi &= cLI\\
  }{
    \Phi &= LI\\
  }
  \emf &= -L\frac{dI}{dt}
\end{align*}
When a steady current $I$ is flowing through an inductor with inductance $L$, the energy stored in the inductor is:
\begin{align*}
  U = \frac{1}{2} L I^2
\end{align*}
\subsubsection{Displacement Current}
In order for the continuity equation to hold under changing magnetic fields, we
must consider another displacement current when using Ampere's law:
\begin{align*}
  \iftoggle{cgs}{
    \ve{J}_D &= \frac{1}{4\pi} \frac{d \ve{E}}{dt} \\
    \ve{\nabla} \times \ve{B} &= \frac{4\pi}{c} (\ve{J} + \ve{J}_D)
  }{
    \ve{J}_D &= \epsilon_0 \frac{d \ve{E}}{dt} \\
    \ve{\nabla} \times \ve{B} &= \mu_0 (\ve{J} + \ve{J}_D)
  }
\end{align*}
\subsection{Electric Circuits}
\section{Oscillations and Waves}
Many questions involve solving linear differential equations. For such
equations, linear combinations of solutions will also be a solution.
\subsection{Oscillations}
\subsubsection{Simple Harmonic Motion}
We have a spring force, $F=-kx$.
\begin{align*}
  \ddot{x}+\omega^2x&=0 \text{, where }\omega=\sqrt{\frac{k}{m}}\\
  x(t)&=A \cos(\omega t+\phi)
\end{align*}
\subsubsection{Damped Oscillators}
In addition to the spring force, we now have a drag force $F_f=-bv$, and the
total force $F=-kx-b\dot{x}$.
\begin{align*}
  \ddot{x} + 2\gamma\dot{x}+\omega^2x=0
\end{align*}
Where $2\gamma=b/m$ and $\omega^2=k/m$. Let $\Omega = \sqrt{\gamma^2-\omega^2}$.
\begin{align*}
  x(t)=e^{-\gamma t}(Ae^{\Omega t}+Be^{-\Omega t})
\end{align*}
\begin{description}
\item [Underdamping] $(\Omega^2<0)$
  \begin{align*}
    x(t)&=e^{-\gamma t}(Ae^{i\tilde{\omega}t}+Be^{-i\tilde{\omega}t})\\
    &=e^{-\gamma t}C\cos(\tilde{\omega}t + \phi)
  \end{align*}
  Where $\tilde{\omega}=\sqrt{\omega^2-\gamma^2}$. The system will oscillate
  with its amplitude decreasing over time. The frequency of oscillations will be
  smaller than in the undamped case.
\item[Overdamping] $(\Omega^2>0)$
  \begin{align*}
    x(t)=Ae^{-(\gamma-\Omega)t}+Be^{-(\gamma+\Omega)t}
  \end{align*}
  The system will not oscillate, and the motion will go to zero for large $t$.
\item[Critical damping] $(\Omega^2=0)$
  We have $\gamma=\omega$, and:
  \begin{align*}
    \ddot{x}+2\gamma\dot{x}+\gamma^2x=0
  \end{align*}
  In this special case, $x=te^{-\gamma t}$ is also a solution:
  \begin{align*}
    x(t)=e^{-\gamma t}(A+Bt)
  \end{align*}
  Systems with critical damping go to zero the quickest.
\end{description}
\subsubsection{Driven Oscillators}
We have to solve differential equations of this form:
\begin{align*}
  \ddot{x}+2\gamma\dot{x}+ax=\sum_{n=1}^{N}{C_ne^{i\omega_nt}}
\end{align*}
We first find particular solutions for each $n$, by guessing solutions of the
form $x_{p_n}(t)=Ae^{i\omega_nt}$:
\begin{align*}
  -A{\omega_n}^2+2iA\gamma\omega_n+Aa=C_n \\
  x_{p_n}(t)=\frac{C_n}{-{\omega_n}^2+2i\gamma\omega_n+a}e^{i\omega_nt}
\end{align*}
Using the superposition principle, the final solution is a linear combination of
the general solution and the particular solutions, with the combination
constants determined by initial conditions.
\subsubsection{Coupled Oscillators}
Normal modes are states of a system where all parts are moving with the same
frequency. General strategy to find normal modes:
\begin{enumerate}
\item Write down the $n$ equations of motions corresponding to the $n$ degrees
  of freedom the system has.
\item Substitute $x_i=A_ie^{i\omega t}$ into the differential equations to get a
  system of linear equations in $A_i$, with $i=1,2,\cdots,n$
\item Non-trivial solutions exist if and only if the determinant of the matrix
  is zero. Solve for $\omega$, and subsequently find $A_i$
\end{enumerate}
The motion of the system can then be decomposed into linear combinations of its
normal modes.
\subsubsection{Small Oscillations}
For an object at a local minimum of a potential well, we can expand $V(x)$ about
the equilibrium point:
\begin{align*}
  V(x)=&V(x_0)+V'(x_0)(x-x_0)\\
  &+ \frac{1}{2!}V''(x_0)(x-x_0)^2+\cdots
\end{align*}
As $V(x_0)$ is an additive constant, and $V'(x_0)=0$ by definition of
equilibrium,
\begin{align*}
  V(x) &\approx \frac{1}{2}V''(x_0)(x-x_0)^2 \\
  F = -\frac{dV}{dx}&=-V''(x_0)(x-x_0) \\
  % m\ddot{x}+V''(x_0)x&=V''(x_0)x_0 \\
  \omega&=\sqrt{\frac{V''(x_0)}{m}}
\end{align*}
\subsection{Wave Equation}
A wave is a disturbance of a continuous medium that propagates with a fixed
shape at constant velocity. In one dimension:
\begin{align*}
  u(z,t)=u(z-vt,0)=f(z-vt)
\end{align*}
All such functions $f$ are the solutions to the wave equation:
\begin{align*}
  \pd{^2 u}{x^2}=\frac{1}{v^2}\pd{^2 u}{t^2}
\end{align*}
Where $v$ is the speed of propagation.

\subsubsection{String with Fixed Ends}
If the equation is subject to the following initial and boundary conditions:
\begin{align*}
  u_x(0, t) &= u_x(L, t) = 0 \\
  u(x, 0) &= f(x) \\
  u_t(x, 0) &= g(x)
\end{align*}
The solution for these conditions is:
\begin{align*}
  u(x, t) &= \sum_{n=1}^{\infty}\sin{\frac{n\pi}{L} x} \; \cdot \\
  &\left( a_n\sin{\frac{n \pi \alpha}{L} t}
    + b_n\cos{\frac{n \pi \alpha}{L} t} \right) \\
  a_n &= \frac{2}{n \pi \alpha}\int_0^L g(x)\sin\frac{n \pi x}{L} dx \\
  b_n &= \frac{2}{L}\int_0^L f(x)\sin\frac{n \pi x}{L} dx
\end{align*}
\subsubsection{D'Alembert's Solution}
For an infinite string, it can be proved that any solution to the wave equation
can be written as a superposition of two waves of velocity $v$, one travelling
to the left, the other travelling to the right. For the initial conditions:
\begin{align*}
  u(x, 0) &= f(x) \\
  u_t(x, 0) &= g(x)
\end{align*}
The solution of the wave equation is:
\begin{align*}
  u(x, t) = \frac{1}{2} &\bigg[ f(x+vt) + f(x-vt)  \\
  &+ \frac{1}{v}\int_{x-vt}^{x+vt}g(x') dx' \bigg]
\end{align*}
\subsubsection{Electromagnetic Waves}
Maxwell's equations in vacuum:
\begin{align*}
  \iftoggle{cgs}{
    \ve{\nabla} \times \ve{E} &= - \frac{1}{c} \frac{d\ve{B}}{dt} \\
  }{
    \ve{\nabla} \times \ve{E} &= - \frac{d\ve{B}}{dt} \\
  }
  \ve{\nabla} \cdot \ve{E} &= 0 \\
  \iftoggle{cgs}{
    \ve{\nabla} \times \ve{B} &= \frac{1}{c} \frac{d\ve{E}}{dt} \\
  }{
    \ve{\nabla} \times \ve{B} &= \mu_0 \epsilon_0 \frac{d\ve{E}}{dt} \\
  }
  \ve{\nabla} \cdot \ve{B} &= 0
\end{align*}
Plugging the equations in and simplifying, we get:
\begin{align*}
  \nabla^2 \ve{B} &= \frac{1}{\iftoggle{cgs}{c^2}{\mu_0 \epsilon_0}}
  \frac{d\ve{B}}{dt} \\
  \nabla^2 \ve{E} &= \frac{1}{\iftoggle{cgs}{c^2}{\mu_0 \epsilon_0}}
  \frac{d\ve{E}}{dt}
\end{align*}
For Maxwell's equations to hold, the $\ve{E}$ and $\ve{B}$ fields and their
direction of propagation are mutually perpendicular. Also, the amplitudes $E_0$
and $B_0$ are related by: $B_0 = \iftoggle{cgs}{}{\frac{1}{c}} E_0$.
\subsubsection{Poynting Vector}
The Poynting vector $\ve{S}$ is defined as:
\begin{align*}
  \ve{S} = \iftoggle{cgs}{\frac{c}{4\pi}}{\frac{1}{\mu_0}} \ve{E} \times \ve{B}
\end{align*}
This vector points in the direction of propagation of the wave, and $\ve{S}
\cdot d\ve{a}$ is the energy per unit time passing through $d\ve{a}$.
\section{Optics}
\subsection{Geometric Optics}
Results from Fermat's principle of least time:
\begin{align*}
  \theta_\text{incidence}&=\theta_\text{reflection} \\
  n_1\sin{\theta_1}&=n_2 \sin{\theta_2}
\end{align*}
Sign convention:
\begin{itemize}
\item Light rays travel from left to right
\item $f$ is positive if surface makes rays more convergent
\item Distances are measured from the surface (left is negative)
\item $s_o$ is negative for real objects
\item $s_i$ is positive for real images
\item $y$ above optical axis is positive
\end{itemize}
\begin{align*}
  \frac{1}{s_{o}} +\frac{1}{f} &=\frac{1}{s_{i}}\\
  M&=\frac{y_i}{y_o}=-\frac{s_i}{s_o}
\end{align*}
For thin lenses and mirrors:
\begin{align*}
  \frac{1}{f} = \frac{2}{R}
\end{align*}
For composite thin lenses:
\begin{align*}
  \frac{1}{f} = (n-1)\left(\frac{1}{R_1} + \frac{1}{R_2}\right)
\end{align*}
Lens formed by interface of two materials with different $n$:
\begin{align*}
  \frac{n_2-n_1}{R} = \frac{n_2}{s_i} + \frac{n_1}{s_o}
\end{align*}
\subsection{Polarization}
For polarized light:
\begin{align*}
  E&=E_0\cos{\theta} \\
  I&=I_0\cos^2{\theta}
\end{align*}
For unpolarized light:
\begin{align*}
  \ave{ I }=I_0 \ave{ \cos^2{\theta} } = \frac{I_0}{2}
\end{align*}
Brewster angle at which all reflected light at an interface is polarized:
\begin{align*}
  \tan{\theta_i}=\frac{n_t}{n_i}
\end{align*}
\subsection{Physical Optics}
Interference is the superposition of wave amplitudes when waves overlap.
\subsubsection{Double Slit:}
Occurs when slits are of negligible width, distance between slits comparable to
wavelength, such that diffraction effects are insignificant. For bright fringes:
\begin{align*}
  d\sin{\theta}&=m\lambda \\
  y_m&=R\frac{m\lambda}{d} \qquad m \in \mathbb{Z}
\end{align*}
For incident medium's refractive index $n_i$, reflection medium's refractive
index $n_r$, if $n_i < n_r$, the reflected wave undergoes a $\frac{\pi}{2}$
phase shift.
\subsubsection{Single Slit:}
Occurs when size of slit is comparable to wavelength. Location of dark fringes
when wavelets at distance $\frac{a}{2}$ destructively interfere:
\begin{align*}
  \sin{\theta}&=\frac{m\lambda}{d} \\
  y_m&=x\frac{m\lambda}{a} \qquad m \in \mathbb{Z}
\end{align*}
\subsubsection{Intensity in Diffraction Patterns}
For double slit interference:
\begin{align*}
  I = I_{\text{max}} \cos^2\left(\frac{\pi d \sin{\theta}}{\lambda} \right)
\end{align*}
For single slit diffraction:
\begin{align*}
  I = I_{\text{max}} \left [\frac{\sin(\pi a \sin \theta / \lambda)}{\pi a \sin
      \theta / \lambda} \right]^2
\end{align*}
Double slit including effects of diffraction:
\begin{align*}
  I = I_{\text{max}} &\cos^2\left(\frac{\pi d \sin{\theta}}{\lambda} \right) \\
  &\cdot \left [\frac{\sin(\pi a \sin \theta / \lambda)}{\pi a \sin \theta /
      \lambda} \right]^2
\end{align*}
\section{Thermodynamics}
If two objects are in thermal equilibrium with a third system, then they are in
equilibrium with each other.
\subsection{Thermal Expansion}
For linear expansion, the change in length is:
\begin{align*}
  \Delta L = \alpha L_0 \Delta T
\end{align*}
Where $\alpha$ is the coefficient of linear expansion. For area expansion, use
approximately $2 \alpha $. For volume expansion, use approximately $3 \alpha$.
\subsection{Kinetic Theory of Gases}
\subsubsection{Ideal Gas Law}
An ideal gas' molecules are treated as non-interacting point particles. For an
ideal gas of $N$ particles at pressure $P$, volume $V$ and temperature $T$:
\begin{align*}
  PV = NK_BT
\end{align*}
For a non-ideal gas, the Van der Waals correction to the ideal gas law is:
\begin{align*}
  \left( P + a \left(\frac{n}{V}\right) ^2 \right) \left( V - bn\right) = nRT
\end{align*}
Where $a$ and $b$ are constants.
\subsubsection{Internal Energy}
Different gases at the same temperature have the same average kinetic
energy. Thus we define temperature of a substance to be its average kinetic
energy. For a monatomic ideal gas:
\begin{align*}
  \frac{1}{2}m\ave{ v^2 } = \frac{3}{2}kT
\end{align*}
\noindent
For a gas molecule with $r$ atoms, its total kinetic energy, center of mass
kinetic energy and internal vibrational/rotational energy are given by:
\begin{align*}
  E_{\text{Total}} &= \frac{3r}{2}kT \\
  E_{\text{COM}} &= \frac{3}{2}kT \\
  E_{\text{Internal}} &= \frac{3(r-1)}{2}kT
\end{align*}
\noindent
The equipartition theorem states that each degree of freedom a molecule has
contributes an extra $\frac{1}{2}KT$ of kinetic energy.
\subsubsection{Maxwell Distribution}
For an ideal gas, the distrubution of its velocities is:
\begin{align*}
  f(v) = 4 \pi v^2 \left( \frac{m}{2 \pi kT} \right)^{\frac{3}{2}}
  e^{-\frac{mv^2}{2kT}}
\end{align*}
From this distribution, we can get the average speed of a particle:
\begin{align*}
  \ave{ v } = \sqrt{\frac{8kT}{\pi m}}
\end{align*}
The most probable velocty is the maximum point of the distribution:
\begin{align*}
  v_{\text{mp}} = \sqrt{\frac{2kT}{m}}
\end{align*}
For any two particles, their average relative speed is:
\begin{align*}
  \ave{ v_{\text{rel}} } = \sqrt{2} \ave{ v } = \sqrt{\frac{16kT}{\pi m}}
\end{align*}
From this, we can get the mean free path of a particle, the average distance a
particle travels before hitting another particle:
\begin{align*}
  l_m = \frac{1}{4 \pi \sqrt{2} r^2 n}
\end{align*}
Where $n$ is the number density of the particle and $r$ is its radius.
\subsubsection{Diffusion}
For a substance undergoing diffusion due to a concentration gradient
$\frac{dc}{dx}$, the diffusive flux $J$ is:
\begin{align*}
  J = D A \frac{dc}{dx}
\end{align*}
\subsection{Heat Teansfer}
For heat transfer through a material with length $l$, area $A$ and thermal
conductivity $K$ between two heat reservoirs $T_1 > T_2$:
\begin{align*}
  \frac{dQ}{dt} = \frac{KA (T_1 - T_2)}{l}
\end{align*}
For a blackbody at temperature $T$ radiating heat away:
\begin{align*}
  \frac{dQ}{dt} = \sigma A T^4
\end{align*}
The heat transfered by changing the temperature of a solid of mass $m$ with heat
capacity $c$ is:
\begin{align*}
  \Delta Q = mc \Delta T
\end{align*}
\subsection{Thermodynamic Processes}
In all the process described below, the heat $Q$ that goes into the gas is
positive, and the work done on the gas $W$ is positive. The first law of
thermodynamics states that the change of internal energy $U$ is:
\begin{align*}
  U &= Q + W \\
  U (\gamma - 1) &= NkT
\end{align*}
Where $\gamma = C_p/C_v$ is the ideal gas constant and $C_v = C_p - k$.
\subsubsection{Isochoric}
In this constant volume process:
\begin{align*}
  W &= 0 \\
  Q &= N C_v \Delta T \\
  U &= Q
\end{align*}
\subsubsection{Isobaric}
In a constant pressure volume expansion from $V_1$ to $V_2$:
\begin{align*}
  W &= P(V_1 - V_2) \\
  Q &= N C_p \Delta T \\
  U &= N C_v \Delta T
\end{align*}
\subsubsection{Isothermal}
For an isothermal expansion from $V_1$ to $V_2$:
\begin{align*}
  W &= NkT \ln \left( \frac{V_1}{V_2} \right) \\
  Q &= - W \\
  U &= 0
\end{align*}
\subsubsection{Adiabatic}
For an adiabatic process,
\begin{align*}
  W &= - \int P dV \\
  Q &= 0 \\
  U &= W
\end{align*}
Integrating the work done, we get the following relation:
\begin{align*}
  PV^{\gamma} = \text{constant}
\end{align*}
\subsection{Heat Engines}
The efficiency of a heat engine that takes in $Q_H$ and gives out $Q_L$ while
doing work $W$, its efficiency is given by:
\begin{align*}
  \eta &= \frac{|W|}{|Q_H|} \\
  &= 1 - \frac{|Q_L|}{|Q_H|}
\end{align*}
The efficiency of a heat pump that uses $W$ to pump $Q_L$ from the col reservoir
is:
\begin{align*}
  \eta &= \frac{|Q_L|}{|W|}
\end{align*}
All reversible engines operating between the same two temperatures have the same
efficiency as a Carnot engine, as you can fit many infinitisimally small Carnot
cycles into any reversible cycle:
\begin{align*}
  \eta_{\text{carnot}} = 1 - \frac{T_L}{T_H}
\end{align*}
\subsection{Second Law of Thermodynamics}
\begin{itemize}
\item A process whose only net result is to take heat from a reservoir and
  convert it to heat is impossible.
\item No heat engine can working between two temperatures $T_1$ and $T_2$ can
  have a higher efficiency than a reversible engine.
\end{itemize}
\subsection{Entropy}
\subsubsection{Macroscopic Definition}
Entropy is the measure of disorder. If heat is added reversibly into a system at
temperature $T$, the increase in entropy in the system is:
\begin{align*}
  dS = \frac{dQ}{T}
\end{align*}
Entropy is a state function that doesn't depend on the path travelled. The total
entropy change in the system and surroundings for a reversible process is
zero. For an irreversible process, the total entropy change is always positive.
\\
\noindent
At $T=0$, $S=0$. This is the third law of thermodynamics.
\subsubsection{Microscopic Definition}
Boltzmann defined entropy of a system by counting the number of
indistinguishable microstates $w$ inside:
\begin{align*}
  S = k \ln w
\end{align*}

\section{Quantum Mechanics}
\subsection{Blackbody Radiation}
Ideal blackbodies have a continuous emission spectrum, with the energy density
$\rho (\lambda, T)$ as a function of wavelength and temperature given by Plank's
distribution:
\begin{align*}
  \rho (\lambda, T) = \frac{8\pi h c}{\lambda^5} \frac{1}{e^{hc/\lambda KT} - 1}
\end{align*}
Integrating over this distribution, we find the total emission power as a
function of temperature:
\begin{align*}
  R(T) = \sigma T^4
\end{align*}
Where $\sigma$ is the Stefan-Boltzmann constant. Finding the wavelength with the peak emission at a given temperature, we arrive at Wien's law:
\begin{align*}
  \lambda_\text{max} T = b
\end{align*}
Where $b$ is a constant.
\subsection{Schr\"{o}dinger's Equation}
$\Psi(x, t)$ is a complex wave function of time and position, the
one-dimensional Schr\"{o}dinger's equation is given by:
\begin{align*}
  i \hbar \pd{\Psi}{t} = - \frac{\hbar^2}{2m} \nabla^2 \Psi + V\Psi
\end{align*}
If we denote the complex conjugate of the wave function to be $\cc{\Psi}$, the
conjugate of Schr\"{o}dinger's equation is:
\begin{align*}
  -i \hbar \pd{\cc{\Psi}}{t} = \frac{\hbar^2}{2m} \nabla^2 \cc{\Psi} -
  V\cc{\Psi}
\end{align*}
At time $t$, the probability of finding a particle from $x=a$ to $x=b$ is:
\begin{align*}
  \int_{a}^{b} |\Psi(\ve{r}, t)|^2 d\ve{r} = \int_{a}^{b}\Psi\cc{\Psi}d\ve{r}
\end{align*}
\subsubsection{Normalization}
All wave functions must be normalized, so that the probability of finding the
particle over all space is 1:
\begin{align*}
  \int_{-\infty}^{\infty} |\Psi(\ve{r},t)|^2 d\ve{r} = 1
\end{align*}
Once a function is normalized, it remains normalized as time evolves:
\begin{align*}
  \frac{d}{dt} \int_{-\infty}^{\infty} \Psi \cc{\Psi} d\ve{r} = 0
\end{align*}
\subsubsection{Expectation Values}
An expectation value of an observed quantity is the average of the measurement
performed on many ``copies'' of the system at the same time.
\begin{align*}
  \ave{x} &= \int_{-\infty}^{\infty} x |\Psi(x,t)|^2 dx \\
  &= \int_{-\infty}^{\infty} \cc{\Psi} x \Psi dx \\
  \ave{p_x} &= m\frac{d\ave{x}}{dt} \\
  &= \int_{-\infty}^{\infty} \cc{\Psi} \left( -i\hbar \pd{}{x} \right) \Psi dx
\end{align*}
In general, the expectation value of any quantity is:
\begin{align*}
  \ave{Q(\ve{r}, \ve{p})} = \int \cc{\Psi} Q\left(\ve{r}, -i\hbar
    \ve{\nabla} \right) \Psi d\ve{r}
\end{align*}
\subsection{Time Independent Solution}
If $V$ is independent of time, we solve Schr\"{o}dinger's equation by separating
variables. Let:
\begin{align*}
  \Psi(x, t) = \psi(x) \phi(t)
\end{align*}
Then the equation can be written as:
\begin{align*}
  i\hbar \psi \pd{\phi}{t} = -\frac{\hbar^2}{2m} \nabla^2 \phi + V \phi \psi \\
  \left( \frac{i\hbar}{\phi} \pd{\phi}{t} \right) + \left(
    \frac{\hbar^2}{2m\psi} \nabla^2 -V(x) \right) = 0
\end{align*}
As the two terms in the equation are independent of each other and they sum to
zero, they must be constant. If we let:
\begin{align*}
  E &= \frac{i\hbar}{\phi} \pd{\phi}{t} \\
  \phi(t) &= e^{-iEt/ \hbar}
\end{align*}
The time independent solution is given by:
\begin{align*}
  -\frac{\hbar^2}{2m}\nabla^2 + V(x) \psi = E \psi\
\end{align*}
If we define the Hamiltonian operator $\operator{H} = -\frac{\hbar^2}{2m}
\nabla^2 + V$,
\begin{align*}
  \operator{H} \psi = E\psi
\end{align*}
The separated solutions can then be combined:
\begin{align*}
  \Psi (\ve{r}, t) = \sum_E C_E(t_0) e^{-iE(t-t_0)/\hbar} \psi_E(\ve{r})
\end{align*}
\subsubsection{Integral Form}
If we integrate the time-independent Schr\"{o}dinger's equation about
$\pm\epsilon$ for small $\epsilon$, the $E\psi$ term disappears and we get:
\begin{align*}
  \left[ \frac{d\psi}{dx} \right]^{+\epsilon}_{-\epsilon} = \frac{2m}{\hbar^2}
  \int^{+\epsilon}_{-\epsilon} V(x) \psi(x) dx
\end{align*}
From this we see that the derivative of $\psi$ is continuous if $V(x)$ is
finite.
\subsection{Momentum Space Wavefunction}
The position space and momentum space representations of a wavefunction can be
interchanged with a Fourier transform:
\begin{align*}
  \psi (x, t) =& \\ \frac{1}{\sqrt{2\pi \hbar}}
  &\int_{-\infty}^\infty e^{iP_xx/\hbar} \, \Phi(P_x, t) \, dP_x \\
  \phi (P_x, t) =& \\ \frac{1}{\sqrt{2\pi \hbar}}
  &\int_{-\infty}^\infty e^{-iP_xx/\hbar} \, \Psi(x, t) \, dx
\end{align*}
\subsection{1-D Examples}
\subsubsection{Free Particle}
For the free particle $V(x) = 0$. We have the general solution:
\begin{align*}
  \psi (x) = Ae^{ikx} + Be^{-ikx}
\end{align*}
After normalization, we have the following solution:
\begin{align*}
  \Psi (x,t) &= \frac{1}{\sqrt{2\pi}} \int_{-\infty}^\infty \phi(k)
  e^{i \left( kx-\frac{\hbar k^2}{2m}t \right)} dk \\
  \phi(k) &= \frac{1}{\sqrt{2\pi}} \int_{-\infty}^\infty \Psi(x, 0) e^{-ikx} dx
\end{align*}
\subsubsection{Infinite Well}
For an infinite well,
\[
  V(x) =
  \begin{cases}
    0 &  |x| < a \\
    \infty & |x| > a
  \end{cases}
\]
The general solution is the same as that of the free particle, but since
$\psi(a) = \psi(-a) = 0$, we have
\[
  \psi_n(x) =
  \begin{cases}
    \frac{1}{\sqrt{a}}\cos \left(\frac{n\pi x}{2a}\right) & n = 1,3,\cdots \\
    \frac{1}{\sqrt{a}}\sin \left(\frac{n\pi x}{2a}\right) & n = 2,4,\cdots \\
  \end{cases}
\]
The energy $E_n$ is proportional to $n^2$:
\begin{align*}
  E_n = \frac{\hbar^2 \pi^2 n^2}{8 m a^2}
\end{align*}

\subsubsection{Finite Well}
For an finite well,
\[
V(x) =
\begin{cases}
  -V_0 & |x| < a \\
  0 & |x| > a
\end{cases}
\]
For $-V_0 \leq E < 0$, we have the following equations:
\begin{align*}
  \frac{d^2 \psi(x)}{dx^2} &+ \alpha^2 \psi(x) = 0, \\
  \alpha &= \sqrt{\frac{2m}{\hbar^2}(V_0 - |E|)} \\
  \frac{d^2 \psi(x)}{dx^2} &- \beta^2 \psi(x) = 0, \\
  \beta &= \sqrt{\frac{2m|E|}{\hbar^2}}
\end{align*}
For $x>0$, the even solutions are:
\[
  \psi(x)=
  \begin{cases}
    A cos(\alpha x) &x \in [0, a] \\
    C e^{-\beta x}  &x > a
  \end{cases}
\]
The odd solutions are:
\[
\psi(x)=
\begin{cases}
  B sin(\alpha x) &x \in [0, a] \\
  C e^{-\beta x}  &x > a
\end{cases}
\]
After solving for boundary conditions, we get:
\begin{align*}
  \alpha \tan(\alpha a) &= \beta \\
  \alpha \cot(\alpha a) &= -\beta \\
\end{align*}

\subsubsection{Harmonic Oscillator}
For a harmonic oscillator,
\begin{align*}
  \operator{H} = -\frac{\hbar^2}{2m} \frac{d^2}{dx^2} + \frac{m}{2}\omega^2 x^2
\end{align*}
Substituting $\xi = \sqrt{\frac{m\omega}{\hbar}}x$, we get solutions:
\begin{align*}
  \psi_n(x) &= \left( \frac{m\omega}{\pi \hbar} \right)^\frac{1}{4} H_n(\xi)
  e^{-\xi^2/2} \frac{1}{\sqrt{2^n n!}} \\
  E_n &= \left(n + \frac{1}{2}\right) \hbar \omega , n = 0,1,\cdots
\end{align*}
Where $H_n$ are the Hermite polynomials:
\begin{alignat*}{3}
  H_0 &= 1 && H_1 = 2\xi \\
  H_2 &= 4\xi^2 - 2 && H_3 = 8\xi^3 - 12\xi \\
  H_4 &= 16\xi^4 - 48\xi^2 + 12 &&
\end{alignat*}

\subsection{Linear Algebra Formalism}
\subsubsection{Postulate 1}
To an ensemble of physical systems, one can assign a wave/state function which contains all the information that can be known of that ensemble $\psi$. This function is in general complex. One can multiply it with an arbitary complex number without changing its physical meaning. For $N$ particles $\psi(\ve{r_1}, \ve{r_2}, \cdots, \ve{r_N})$, $\cc{\psi} \psi$ gives the probability density that \#1 is at $\ve{r_1}$, \#2 is at $\ve{r_2}$, etc. We represent the state vector as $\ket{\psi}$ and the adjoint vector as $\bra{\psi}$:
\begin{align*}
  \braket{\psi_1}{\psi_2} &= \int \cc{\psi_1} \psi_2 d\ve{r} = \cc{\braket{\psi_2}{\psi_1}}\\
  \braket{\psi_1}{c\psi_2} &= c \braket{\psi_1}{\psi_2} \\
  \braket{c\psi_1}{\psi_2} &= \cc{c} \braket{\psi_1}{\psi_2} \\
  \braket{\psi_3}{\psi_1 + \psi_2} &= \braket{\psi_3}{\psi_1} + \braket{\psi_3}{\psi_2} \\
  \braket{\psi}{\psi} &= 1\\
\end{align*}
$\psi_1$ and $\psi_2$ are orthogonal if and only if $\braket{\psi_1}{\psi_2} = 0$.

\subsubsection{Postulate 2}
The principle of superposition: if $\psi_1$ and $\psi_2$ are solutions to the Schr\"{o}dinger's equation, so is $\psi = c_1 \psi_1 + c_2 \psi_2$.

\subsubsection{Postulate 3}
Every dynamical variable is associated with a linear operator. $\mathcal{A} = A(\ve{r_1}, \cdots, \ve{r_n}, \ve{p_1}, \cdots, \ve{p_n}, t) = \operator{A}(\ve{r_1}, \cdots, \ve{r_n}, i\hbar \ve{\nabla_1}, \cdots, i\hbar \ve{\nabla_n}, t)$.

\subsubsection{Postulate 4}
The only result of a precise measurement is one of the eigenvalues of the associated linear operator $\operator{A}$:
\begin{align*}
  \operator{A} \psi_n = a_n \psi_n
\end{align*}
To have real eigenvalues, $\operator{A}$ must be Hermitian:
\begin{align*}
  \braket{\psi_n}{\operator{A} \psi_n} = \braket{\operator{A} \psi_n}{\psi_n} = a_n\braket{\psi_n}{\psi_n}
\end{align*}

\subsubsection{Postulate 5}
If a system of measurements is made of the dynamical variable $A$ on an ensemble of identical systems each described by the same $\ket{\psi}$, the average value is given by:
\begin{align*}
  \ave{\operator{A}} = \frac{\braket{\psi}{\operator{A} \psi}}{\braket{\psi}{\psi}}
\end{align*}
\end{multicols*}
\end{document}
