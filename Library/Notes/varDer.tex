\documentclass[a4paper, 12pt]{article}
\usepackage[margin=2cm]{geometry}
\usepackage{amssymb, amsmath, graphicx, tabularx, booktabs, subfiles, amsthm,enumitem,xcolor,appendix}
 

\newtheorem{cse}{Case}
\newcommand{\ans}[1]{\textcolor{red}{#1}}

\title{Laplacian Operator in Transformed Coordinates }
\author{Rajesh Kumar}

\begin{document}
\maketitle
%------------Body-Starts--------------------
\section*{2D}
Let's denote the inverse transformation from \( (u, v) \) coordinates to \( (x, y) \) coordinates as follows:

\[ x = X(u, v) \]
\[ y = Y(u, v) \]

Now, we want to express the partial derivatives needed for the Laplacian operator in terms of \( u \) and \( v \). The chain rule gives us the partial derivatives with respect to \( u \) and \( v \):

\[ \frac{\partial}{\partial x} = \frac{\partial u}{\partial x}\frac{\partial}{\partial u} + \frac{\partial v}{\partial x}\frac{\partial}{\partial v} \]
\[ \frac{\partial}{\partial y} = \frac{\partial u}{\partial y}\frac{\partial}{\partial u} + \frac{\partial v}{\partial y}\frac{\partial}{\partial v} \]

Now, express the Laplacian operator in terms of \( u \) and \( v \):

\[ \nabla^2 = \frac{\partial^2}{\partial x^2} + \frac{\partial^2}{\partial y^2} \]

Substitute the expressions for \( \frac{\partial}{\partial x} \) and \( \frac{\partial}{\partial y} \) into the Laplacian operator:

\[ \nabla^2 = \left(\frac{\partial u}{\partial x}\right)^2\frac{\partial^2}{\partial u^2} + 2\frac{\partial u}{\partial x}\frac{\partial v}{\partial x}\frac{\partial^2}{\partial u \partial v} + \left(\frac{\partial v}{\partial x}\right)^2\frac{\partial^2}{\partial v^2} \]
\[ + \left(\frac{\partial u}{\partial y}\right)^2\frac{\partial^2}{\partial u^2} + 2\frac{\partial u}{\partial y}\frac{\partial v}{\partial y}\frac{\partial^2}{\partial u \partial v} + \left(\frac{\partial v}{\partial y}\right)^2\frac{\partial^2}{\partial v^2} \]

This expresses the Laplacian operator in terms of \( u \) and \( v \).

\section*{3D}
Given that \( x \), \( y \), and \( z \) are functions of \( u, v, \) and \( w \), we can express the second-order partial derivatives with respect to \( x, y, \) and \( z \) in terms of \( u, v, \) and \( w \) using the chain rule.

Let's denote \( x(u, v, w) \), \( y(u, v, w) \), and \( z(u, v, w) \) as the functions that describe the coordinate transformation. Then, the second-order partial derivatives are:

\[ \frac{\partial^2}{\partial x^2} = \frac{\partial}{\partial x}\left(\frac{\partial}{\partial x}\right) \]
\[ \frac{\partial^2}{\partial y^2} = \frac{\partial}{\partial y}\left(\frac{\partial}{\partial y}\right) \]
\[ \frac{\partial^2}{\partial z^2} = \frac{\partial}{\partial z}\left(\frac{\partial}{\partial z}\right) \]

Using the chain rule, these derivatives can be expressed in terms of \( u, v, \) and \( w \). Let's calculate each term:

\[ \frac{\partial}{\partial x} = \frac{\partial u}{\partial x}\frac{\partial}{\partial u} + \frac{\partial v}{\partial x}\frac{\partial}{\partial v} + \frac{\partial w}{\partial x}\frac{\partial}{\partial w} \]

\[ \frac{\partial}{\partial y} = \frac{\partial u}{\partial y}\frac{\partial}{\partial u} + \frac{\partial v}{\partial y}\frac{\partial}{\partial v} + \frac{\partial w}{\partial y}\frac{\partial}{\partial w} \]

\[ \frac{\partial}{\partial z} = \frac{\partial u}{\partial z}\frac{\partial}{\partial u} + \frac{\partial v}{\partial z}\frac{\partial}{\partial v} + \frac{\partial w}{\partial z}\frac{\partial}{\partial w} \]

Now, substitute these expressions into the expressions for the second-order partial derivatives and simplify:

\[ \frac{\partial^2}{\partial x^2} = \left(\frac{\partial u}{\partial x}\right)^2\frac{\partial^2}{\partial u^2} + \left(\frac{\partial v}{\partial x}\right)^2\frac{\partial^2}{\partial v^2} + \left(\frac{\partial w}{\partial x}\right)^2\frac{\partial^2}{\partial w^2} + 2\frac{\partial u}{\partial x}\frac{\partial v}{\partial x}\frac{\partial^2}{\partial u\partial v} + 2\frac{\partial u}{\partial x}\frac{\partial w}{\partial x}\frac{\partial^2}{\partial u\partial w} + 2\frac{\partial v}{\partial x}\frac{\partial w}{\partial x}\frac{\partial^2}{\partial v\partial w} \]



The complete expression for the Laplacian operator \( \nabla^2 \) after substituting the expressions for the second-order partial derivatives in terms of \( u, v, w \), we get:

\[ \nabla^2 = \left(\frac{\partial u}{\partial x}\right)^2\frac{\partial^2}{\partial u^2} + \left(\frac{\partial v}{\partial x}\right)^2\frac{\partial^2}{\partial v^2} + \left(\frac{\partial w}{\partial x}\right)^2\frac{\partial^2}{\partial w^2} \]
\[ + 2\frac{\partial u}{\partial x}\frac{\partial v}{\partial x}\frac{\partial^2}{\partial u\partial v} + 2\frac{\partial u}{\partial x}\frac{\partial w}{\partial x}\frac{\partial^2}{\partial u\partial w} + 2\frac{\partial v}{\partial x}\frac{\partial w}{\partial x}\frac{\partial^2}{\partial v\partial w} \]

\[ + \left(\frac{\partial u}{\partial y}\right)^2\frac{\partial^2}{\partial u^2} + \left(\frac{\partial v}{\partial y}\right)^2\frac{\partial^2}{\partial v^2} + \left(\frac{\partial w}{\partial y}\right)^2\frac{\partial^2}{\partial w^2} \]
\[ + 2\frac{\partial u}{\partial y}\frac{\partial v}{\partial y}\frac{\partial^2}{\partial u\partial v} + 2\frac{\partial u}{\partial y}\frac{\partial w}{\partial y}\frac{\partial^2}{\partial u\partial w} + 2\frac{\partial v}{\partial y}\frac{\partial w}{\partial y}\frac{\partial^2}{\partial v\partial w} \]

\[ + \left(\frac{\partial u}{\partial z}\right)^2\frac{\partial^2}{\partial u^2} + \left(\frac{\partial v}{\partial z}\right)^2\frac{\partial^2}{\partial v^2} + \left(\frac{\partial w}{\partial z}\right)^2\frac{\partial^2}{\partial w^2} \]
\[ + 2\frac{\partial u}{\partial z}\frac{\partial v}{\partial z}\frac{\partial^2}{\partial u\partial v} + 2\frac{\partial u}{\partial z}\frac{\partial w}{\partial z}\frac{\partial^2}{\partial u\partial w} + 2\frac{\partial v}{\partial z}\frac{\partial w}{\partial z}\frac{\partial^2}{\partial v\partial w} \]

This is the complete expression for \( \nabla^2 \) in terms of the new coordinates \( u, v, w \) when \( x, y, z \) are functions of \( u, v, w \).






%------------Body-Ends--------------------
%\bibliography{ref.bib}
%\bibliographystyle{IEEEtran}
\end{document}
