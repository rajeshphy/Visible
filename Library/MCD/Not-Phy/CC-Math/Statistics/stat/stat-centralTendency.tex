
\title{Measures of Central Tendency}
\author{Rajesh Kumar}
\date{}

\begin{frame}
  \titlepage
\end{frame}

\begin{frame}{Measures of Central Tendency}
  \begin{itemize}
    \item Measures of central tendency are statistical measures that indicate the center or average of a dataset.
    \item Common measures include the mean, median, and mode.
  \end{itemize}
\end{frame}

\begin{frame}{Introduction}
  \begin{itemize}
    \item These measures provide a way to summarize and describe the central value of a set of data.
    \item They are useful for understanding the typical or representative value in a distribution.
  \end{itemize}
\end{frame}

\begin{frame}{Where it is Used in Real Life}
  \begin{itemize}
    \item \textbf{Education:} Analyzing average exam scores to understand class performance.
    \item \textbf{Economics:} Calculating the average income to represent the central income level.
    \item \textbf{Healthcare:} Determining the average recovery time for a specific medical treatment.
  \end{itemize}
\end{frame}

\begin{frame}{Common Measures of Central Tendency}
  \begin{itemize}
    \item \textbf{Mean:} The sum of all values divided by the number of values.
    \item \textbf{Median:} The middle value when the data is ordered.
    \item \textbf{Mode:} The most frequently occurring value(s) in the dataset.
  \end{itemize}
\end{frame}

\begin{frame}{Calculation of Measures}
  \begin{itemize}
    \item \textbf{Mean:} \( \bar{x} = \frac{\sum_{i=1}^{n} x_i}{n} \)
    \item \textbf{Median:} Depends on whether the number of observations is odd or even.
    \item \textbf{Mode:} The value(s) with the highest frequency in the dataset.
  \end{itemize}
\end{frame}

\begin{frame}{Worked Out Example}
  \textbf{Example:} Calculate the mean, median, and mode for the dataset: [12, 15, 18, 22, 22, 25, 28, 30, 35].

  \textbf{Solution:}
  \begin{itemize}
    \item \textbf{Mean:} \( \bar{x} = \frac{12 + 15 + 18 + 22 + 22 + 25 + 28 + 30 + 35}{9} \)
    \item \textbf{Median:} Since there are 9 observations, the median is the 5th value (22).
    \item \textbf{Mode:} 22 is the mode as it appears more frequently than other values.
  \end{itemize}
\end{frame}

\begin{frame}{Reasoning for Measures of Central Tendency Solution}
  \begin{itemize}
    \item Measures of central tendency provide a summary of the center of the data.
    \item For the given example, we calculated the mean, median, and mode to represent the central tendency.
  \end{itemize}
\end{frame}

\begin{frame}{Exercise for Students}
  \begin{enumerate}
    \item Calculate the mean, median, and mode for the dataset: [18, 20, 22, 22, 25, 25, 28, 30].
    \item Discuss the situations in which each measure (mean, median, mode) is most appropriate to use.
    \item Analyze the central tendency of the salaries in a company.
  \end{enumerate}
\end{frame}

