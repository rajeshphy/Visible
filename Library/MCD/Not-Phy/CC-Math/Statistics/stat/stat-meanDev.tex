
\title{Mean Deviation as a Measure of Dispersion}
\author{Rajesh Kumar}
\date{}

\begin{frame}
  \titlepage
\end{frame}

\begin{frame}{Mean Deviation as a Measure of Dispersion}
  \begin{itemize}
    \item Mean Deviation is a measure of the spread or dispersion of a dataset.
    \item It quantifies the average distance between each data point and the mean of the dataset.
  \end{itemize}
\end{frame}

\begin{frame}{Introduction}
  \begin{itemize}
    \item Mean Deviation is calculated by finding the average of the absolute differences between each data point and the mean.
    \item It provides a measure of the variability of data points from the central value.
  \end{itemize}
\end{frame}

\begin{frame}{Where it is Used in Real Life}
  \begin{itemize}
    \item \textbf{Finance:} Analyzing the variability of stock prices around the average.
    \item \textbf{Education:} Assessing the spread of student scores around the class average.
    \item \textbf{Healthcare:} Examining the variability of patient recovery times in a treatment study.
  \end{itemize}
\end{frame}

\begin{frame}{Calculation of Mean Deviation}
  \begin{itemize}
    \item The Mean Deviation (\(MD\)) is calculated using the formula:
    \[ MD = \frac{\sum_{i=1}^{n} |x_i - \bar{x}|}{n} \]
    \item Where \(x_i\) represents each individual value, \(\bar{x}\) is the mean, \(\sum\) denotes the sum, and \(n\) is the number of values.
  \end{itemize}
\end{frame}

\begin{frame}{Worked Out Example}
  \textbf{Example:} Calculate the Mean Deviation for the dataset: [12, 15, 18, 22, 22, 25, 28, 30, 35].

  \textbf{Solution:}
  \begin{itemize}
    \item \textbf{Mean Deviation:} \[ MD = \frac{|12-23.33| + |15-23.33| + \ldots + |35-23.33|}{9} \]
  \end{itemize}
\end{frame}

\begin{frame}{Reasoning for Mean Deviation Solution}
  \begin{itemize}
    \item Mean Deviation provides insight into the average spread of data points from the mean.
    \item For the given example, we calculated the Mean Deviation to quantify the dispersion in the dataset.
  \end{itemize}
\end{frame}

\begin{frame}{Exercise for Students}
  \begin{enumerate}
    \item Calculate the Mean Deviation for the dataset: [18, 20, 22, 22, 25, 25, 28, 30].
    \item Discuss situations where Mean Deviation is a useful measure of dispersion.
    \item Analyze the Mean Deviation of a set of test scores in a class.
  \end{enumerate}
\end{frame}
