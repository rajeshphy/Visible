
\title{Standard Deviation as a Measure of Dispersion}
\author{Rajesh Kumar}
\date{}

\begin{frame}
  \titlepage
\end{frame}

\begin{frame}{Standard Deviation as a Measure of Dispersion}
  \begin{itemize}
    \item Standard Deviation is a measure of the spread or dispersion of a dataset.
    \item It quantifies the average deviation of data points from the mean.
  \end{itemize}
\end{frame}

\begin{frame}{Introduction}
  \begin{itemize}
    \item Standard Deviation is calculated by taking the square root of the variance.
    \item It provides a more robust measure of dispersion, giving higher weights to larger deviations.
  \end{itemize}
\end{frame}

\begin{frame}{Where it is Used in Real Life}
  \begin{itemize}
    \item \textbf{Finance:} Assessing the risk and volatility of investment returns.
    \item \textbf{Education:} Analyzing the variability of student scores in a standardized test.
    \item \textbf{Healthcare:} Examining the spread of patient recovery times in a clinical trial.
  \end{itemize}
\end{frame}

\begin{frame}{Calculation of Standard Deviation}
  \begin{itemize}
    \item The Standard Deviation (\(\sigma\) for population, \(s\) for sample) is calculated using the formula:
    \[ \sigma = \sqrt{\frac{\sum_{i=1}^{n} (x_i - \bar{x})^2}{n}} \]
    \[ s = \sqrt{\frac{\sum_{i=1}^{n-1} (x_i - \bar{x})^2}{n-1}} \]
    \item Where \(x_i\) represents each individual value, \(\bar{x}\) is the mean, \(\sum\) denotes the sum, and \(n\) is the number of values.
  \end{itemize}
\end{frame}

\begin{frame}{Worked Out Example}
  \textbf{Example:} Calculate the Standard Deviation for the dataset: [12, 15, 18, 22, 22, 25, 28, 30, 35].

  \textbf{Solution:}
  \begin{itemize}
    \item \textbf{Standard Deviation:} \[ \sigma = \sqrt{\frac{(12-23.33)^2 + (15-23.33)^2 + \ldots + (35-23.33)^2}{9}} \]
  \end{itemize}
\end{frame}

\begin{frame}{Reasoning for Standard Deviation Solution}
  \begin{itemize}
    \item Standard Deviation provides a measure of dispersion that considers the magnitude of deviations.
    \item For the given example, we calculated the Standard Deviation to quantify the spread of the dataset.
  \end{itemize}
\end{frame}

\begin{frame}{Exercise for Students}
  \begin{enumerate}
    \item Calculate the Standard Deviation for the dataset: [18, 20, 22, 22, 25, 25, 28, 30].
    \item Discuss situations where Standard Deviation is a useful measure of dispersion.
    \item Analyze the Standard Deviation of a set of temperatures recorded over a month.
  \end{enumerate}
\end{frame}
