\title{Median as a Measure of Central Tendency}
\author{Rajesh Kumar}
\date{}

\begin{frame}
  \titlepage
\end{frame}

\begin{frame}{Median as a Measure of Central Tendency}
  \begin{itemize}
    \item The median is a statistical measure of central tendency that represents the middle value in a sorted dataset.
    \item It is less sensitive to extreme values (outliers) compared to the mean.
  \end{itemize}
\end{frame}

\begin{frame}{Introduction}
  \begin{itemize}
    \item The median is particularly useful when dealing with skewed or non-normally distributed data.
    \item It is the value below which 50\% of the data falls.
  \end{itemize}
\end{frame}

\begin{frame}{Where it is Used in Real Life}
  \begin{itemize}
    \item \textbf{Housing:} Analyzing the median home price in a neighborhood.
    \item \textbf{Education:} Understanding the median score on a standardized test.
    \item \textbf{Healthcare:} Examining the median age of patients in a medical study.
  \end{itemize}
\end{frame}

\begin{frame}{Calculation of Median}
  \begin{itemize}
    \item The median is determined differently for datasets with an odd or even number of values:
    \begin{itemize}
      \item If odd, it is the middle value.
      \item If even, it is the average of the two middle values.
    \end{itemize}
  \end{itemize}
\end{frame}

\begin{frame}{Worked Out Example}
  \textbf{Example:} Calculate the median for the dataset: \([12, 15, 18, 22, 22, 25, 28, 30, 35]\).

  \textbf{Solution:}
  \begin{itemize}
    \item Since there are 9 observations, the median is the 5th value when the data is sorted (22).
  \end{itemize}
\end{frame}

\begin{frame}{Reasoning for Median Solution}
  \begin{itemize}
    \item The median provides a central value that is not influenced by extreme values.
    \item For the given example, we calculated the median to represent the middle value of the dataset.
  \end{itemize}
\end{frame}

\begin{frame}{Exercise for Students}
  \begin{enumerate}
    \item Calculate the median for the dataset: \([18, 20, 22, 22, 25, 25, 28, 30]\).
    \item Discuss situations where the median may be a better measure of central tendency than the mean.
    \item Analyze the median income of a group of individuals.
  \end{enumerate}
\end{frame}
