
\title{Mean as a Measure of Central Tendency}
\author{Rajesh Kumar}
\date{}

\begin{frame}
  \titlepage
\end{frame}

\begin{frame}{Mean as a Measure of Central Tendency}
  \begin{itemize}
    \item The mean is a statistical measure of central tendency that represents the average of a set of values.
    \item It is calculated by summing up all values and dividing by the number of values.
  \end{itemize}
\end{frame}

\begin{frame}{Introduction}
  \begin{itemize}
    \item The mean is a widely used measure, providing a central value around which the data is distributed.
    \item It is sensitive to extreme values, also known as outliers.
  \end{itemize}
\end{frame}

\begin{frame}{Where it is Used in Real Life}
  \begin{itemize}
    \item \textbf{Education:} Calculating the average score in a class.
    \item \textbf{Finance:} Determining the average income or expenses.
    \item \textbf{Sports:} Computing the average performance of a player over a season.
  \end{itemize}
\end{frame}

\begin{frame}{Calculation of Mean}
  \begin{itemize}
    \item The mean (\(\bar{x}\)) is calculated using the formula:
    \[ \bar{x} = \frac{\sum_{i=1}^{n} x_i}{n} \]
    \item Where \(x_i\) represents each individual value, \(\sum\) denotes the sum, and \(n\) is the number of values.
  \end{itemize}
\end{frame}

\begin{frame}{Worked Out Example}
  \textbf{Example:} Calculate the mean for the dataset: [12, 15, 18, 22, 22, 25, 28, 30, 35].

  \textbf{Solution:}
  \begin{itemize}
    \item \textbf{Mean:} \( \bar{x} = \frac{12 + 15 + 18 + 22 + 22 + 25 + 28 + 30 + 35}{9} \)
  \end{itemize}
\end{frame}

\begin{frame}{Reasoning for Mean Solution}
  \begin{itemize}
    \item The mean is a measure that represents the average of the dataset.
    \item For the given example, we calculated the mean to provide a central value for the data.
  \end{itemize}
\end{frame}

\begin{frame}{Exercise for Students}
  \begin{enumerate}
    \item Calculate the mean for the dataset: [18, 20, 22, 22, 25, 25, 28, 30].
    \item Discuss situations where the mean may not accurately represent the central tendency of the data.
    \item Analyze the mean income of a group of individuals.
  \end{enumerate}
\end{frame}
