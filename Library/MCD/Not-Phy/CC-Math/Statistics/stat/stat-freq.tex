\title{Frequency Distribution}
\author{Rajesh Kumar}
\date{}

\begin{frame}
  \titlepage
\end{frame}

\begin{frame}{Frequency Distribution}
  \begin{itemize}
    \item Frequency distribution is a tabular or graphical representation of data that shows the frequency of different outcomes in a dataset.
    \item It helps in organizing and summarizing data for better analysis and interpretation.
  \end{itemize}
\end{frame}

\begin{frame}{Introduction}
  \begin{itemize}
    \item Frequency distribution provides a way to understand the distribution and pattern of values in a dataset.
    \item It involves grouping data into intervals (bins) and counting the number of observations in each interval.
  \end{itemize}
\end{frame}

\begin{frame}{Where it is Used in Real Life}
  \begin{itemize}
    \item \textbf{Education:} Analyzing student performance scores across different grade ranges.
    \item \textbf{Business:} Examining the distribution of customer ages in a market.
    \item \textbf{Healthcare:} Studying the frequency of blood pressure levels in a population.
  \end{itemize}
\end{frame}

\begin{frame}{Steps to Create Frequency Distribution}
  \begin{enumerate}
    \item \textbf{Collect Data:} Gather the dataset you want to analyze.
    \item \textbf{Determine Number of Intervals:} Decide on the number of intervals or bins.
    \item \textbf{Calculate Range:} Find the range of the data (difference between the maximum and minimum values).
    \item \textbf{Calculate Interval Width:} Divide the range by the number of intervals to determine the width.
    \item \textbf{Create Intervals:} Define intervals and count the frequency of data points in each interval.
    \item \textbf{Construct the Table or Graph:} Present the frequency distribution in a table or graph.
  \end{enumerate}
\end{frame}

\begin{frame}{Worked Out Example}
  \textbf{Example:} Create a frequency distribution for the following dataset: \([15, 22, 18, 25, 30, 18, 22, 27, 18, 20, 25, 22]\).

  \textbf{Solution:}
  \begin{enumerate}
    \item \textbf{Sort the Data:} \([15, 18, 18, 18, 20, 22, 22, 22, 25, 25, 27, 30]\).
    \item \textbf{Determine Intervals:} Let's use intervals of size 5 (e.g., 15-19, 20-24, etc.).
    \item \textbf{Count Frequency:} Count the number of data points in each interval.
    \item \textbf{Construct Frequency Distribution:} Present the results in a table or graph.
  \end{enumerate}
\end{frame}

\begin{frame}{Reasoning for Frequency Distribution Solution}
  \begin{itemize}
    \item Frequency distribution helps in summarizing and organizing data.
    \item For the given example, we sorted the data, determined intervals, counted frequencies, and presented the results.
  \end{itemize}
\end{frame}

\begin{frame}{Exercise for Students}
  \begin{enumerate}
    \item Create a frequency distribution for the ages of students in a class.
    \item Analyze the distribution of weights in a sample of individuals.
    \item Discuss a real-life scenario where understanding frequency distribution is important.
  \end{enumerate}
\end{frame}

