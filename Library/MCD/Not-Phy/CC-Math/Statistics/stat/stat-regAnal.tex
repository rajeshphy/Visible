\title{Correlation and Regression Analysis}
\author{Rajesh Kumar}
\date{}

\begin{frame}
  \titlepage
\end{frame}

\begin{frame}{Correlation and Regression Analysis}
  \begin{itemize}
    \item Correlation and Regression Analysis are statistical techniques used to examine relationships between variables.
    \item They help understand the strength and direction of associations between two or more variables.
  \end{itemize}
\end{frame}

\begin{frame}{Correlation Analysis}
  \begin{itemize}
    \item Correlation measures the strength and direction of a linear relationship between two variables.
    \item It is represented by the correlation coefficient (\(r\)), ranging from -1 to 1.
    \item Positive values indicate a positive correlation, negative values indicate a negative correlation, and 0 indicates no correlation.
  \end{itemize}
\end{frame}

\begin{frame}{Introduction to Correlation Analysis}
  \begin{itemize}
    \item Correlation does not imply causation; it only shows the degree of association.
    \item Scatter plots are often used to visualize the relationship between variables.
  \end{itemize}
\end{frame}

\begin{frame}{Where it is Used in Real Life}
  \begin{itemize}
    \item \textbf{Finance:} Examining the correlation between stock prices.
    \item \textbf{Healthcare:} Analyzing the correlation between exercise and heart health.
    \item \textbf{Education:} Investigating the correlation between study hours and exam scores.
  \end{itemize}
\end{frame}

\begin{frame}{Calculation of Correlation Coefficient}
  \begin{itemize}
    \item The correlation coefficient (\(r\)) is calculated using the formula:
    \[ r = \frac{\sum{(x_i - \bar{x})(y_i - \bar{y})}}{\sqrt{\sum{(x_i - \bar{x})^2} \cdot \sum{(y_i - \bar{y})^2}}} \]
    \item Where \(x_i\) and \(y_i\) are individual data points, \(\bar{x}\) and \(\bar{y}\) are means, and \(\sum\) denotes the sum.
  \end{itemize}
\end{frame}

\begin{frame}{Worked Out Example - Correlation}
  \textbf{Example:} Calculate the correlation coefficient (\(r\)) for two variables: \(X = [10, 15, 20, 25, 30]\) and \(Y = [25, 20, 15, 10, 5]\).

  \textbf{Solution:}
  \begin{itemize}
    \item \textbf{Correlation Coefficient:} \[ r = \frac{(10-20)(25-15) + (15-20)(20-15) + \ldots + (30-20)(5-15)}{\sqrt{(10-20)^2 + (15-20)^2 + \ldots + (30-20)^2} \cdot \sqrt{(25-15)^2 + (20-15)^2 + \ldots + (5-15)^2}} \]
  \end{itemize}
\end{frame}

\begin{frame}{Reasoning for Correlation Analysis}
  \begin{itemize}
    \item Correlation analysis helps us understand the strength and direction of the linear relationship between variables.
    \item For the given example, we calculated the correlation coefficient (\(r\)) to quantify the association between two variables.
  \end{itemize}
\end{frame}

\begin{frame}{Regression Analysis}
  \begin{itemize}
    \item Regression analysis explores the relationship between a dependent variable (\(Y\)) and one or more independent variables (\(X\)).
    \item It aims to model the nature of the relationship and make predictions.
  \end{itemize}
\end{frame}

\begin{frame}{Introduction to Regression Analysis}
  \begin{itemize}
    \item The simplest form is simple linear regression, which involves one independent variable.
    \item The regression equation is represented as \(Y = \beta_0 + \beta_1X + \varepsilon\), where \(\beta_0\) and \(\beta_1\) are coefficients, and \(\varepsilon\) is the error term.
  \end{itemize}
\end{frame}

\begin{frame}{Where it is Used in Real Life}
  \begin{itemize}
    \item \textbf{Economics:} Modeling the relationship between income and spending.
    \item \textbf{Marketing:} Predicting sales based on advertising spending.
    \item \textbf{Healthcare:} Estimating the impact of a variable on patient outcomes.
  \end{itemize}
\end{frame}

\begin{frame}{Calculation of Regression Coefficients}
  \begin{itemize}
    \item The regression coefficients (\(\beta_0\) and \(\beta_1\)) are calculated using the formulas:
    \[ \beta_1 = \frac{\sum{(x_i - \bar{x})(y_i - \bar{y})}}{\sum{(x_i - \bar{x})^2}} \]
    \[ \beta_0 = \bar{y} - \beta_1\bar{x} \]
  \end{itemize}
\end{frame}

\begin{frame}{Worked Out Example - Regression}
  \textbf{Example:} Perform simple linear regression for the variables \(X = [10, 15, 20, 25, 30]\) and \(Y = [25, 20, 15, 10, 5]\).

  \textbf{Solution:}
  \begin{itemize}
    \item \textbf{Regression Coefficients:} \[ \beta_1 = \frac{(10-20)(25-15) + (15-20)(20-15) + \ldots + (30-20)(5-15)}{(10-20)^2 + (15-20)^2 + \ldots + (30-20)^2} \]
    \[ \beta_0 = \bar{y} - \beta_1\bar{x} \]
  \end{itemize}
\end{frame}

\begin{frame}{Reasoning for Regression Analysis}
  \begin{itemize}
    \item Regression analysis helps us model and predict the relationship between variables.
    \item For the given example, we calculated the regression coefficients (\(\beta_0\) and \(\beta_1\)) to represent the relationship between two variables.
  \end{itemize}
\end{frame}

\begin{frame}{Exercise for Students}
  \begin{enumerate}
    \item Calculate the correlation coefficient (\(r\)) for two variables in a given dataset.
    \item Perform simple linear regression for a set of variables and interpret the results.
    \item Discuss scenarios where correlation and regression analysis are valuable in making predictions.
  \end{enumerate}
\end{frame}
