\title{Mode as a Measure of Central Tendency}
\author{Rajesh Kumar}
\date{}

\begin{frame}
  \titlepage
\end{frame}

\begin{frame}{Mode as a Measure of Central Tendency}
  \begin{itemize}
    \item The mode is a statistical measure of central tendency that represents the most frequently occurring value(s) in a dataset.
    \item It is particularly useful for categorical or discrete data.
  \end{itemize}
\end{frame}

\begin{frame}{Introduction}
  \begin{itemize}
    \item Unlike the mean and median, the mode can be applied to both numerical and categorical data.
    \item A dataset may have one mode (unimodal), two modes (bimodal), or more (multimodal).
  \end{itemize}
\end{frame}

\begin{frame}{Where it is Used in Real Life}
  \begin{itemize}
    \item \textbf{Education:} Identifying the most common grade in a class.
    \item \textbf{Business:} Determining the most popular product in a product line.
    \item \textbf{Healthcare:} Analyzing the most prevalent health condition in a population.
  \end{itemize}
\end{frame}

\begin{frame}{Calculation of Mode}
  \begin{itemize}
    \item The mode is simply the value(s) with the highest frequency in the dataset.
    \item For continuous data, it can be more challenging, as there may be no clear mode or multiple modes.
  \end{itemize}
\end{frame}

\begin{frame}{Worked Out Example}
  \textbf{Example:} Find the mode for the dataset: \([12, 15, 18, 22, 22, 25, 28, 30, 35]\).

  \textbf{Solution:}
  \begin{itemize}
    \item The mode is 22, as it appears more frequently than any other value in the dataset.
  \end{itemize}
\end{frame}

\begin{frame}{Reasoning for Mode Solution}
  \begin{itemize}
    \item The mode identifies the most common value(s) in a dataset.
    \item For the given example, we found the mode to represent the most frequently occurring value.
  \end{itemize}
\end{frame}

\begin{frame}{Exercise for Students}
  \begin{enumerate}
    \item Find the mode for the dataset: \([18, 20, 22, 22, 25, 25, 28, 30]\).
    \item Discuss situations where the mode may be a suitable measure of central tendency.
    \item Analyze the mode of the most frequently purchased item in a store.
  \end{enumerate}
\end{frame}
