
\title{Conditional Probability}
\author{Rajesh Kumar}
\date{}

\begin{frame}
  \titlepage
\end{frame}

\begin{frame}{Definition}
  \begin{itemize}
    \item \textbf{Conditional Probability:} The probability of an event occurring given that another event has already occurred, expressed as \( P(A|B) \).
  \end{itemize}
\end{frame}

\begin{frame}{Introduction}
  \begin{itemize}
    \item Conditional probability provides a way to calculate the likelihood of an event under the condition that another event has happened.
    \item It is an essential concept in probability theory and has applications in various fields, including statistics and decision-making.
  \end{itemize}
\end{frame}

\begin{frame}{Where it is Used in Real Life}
  \begin{itemize}
    \item \textbf{Medical Diagnosis:} The probability of having a certain medical condition given the results of a diagnostic test.
    \item \textbf{Insurance:} Calculating the probability of an insurance claim given certain risk factors.
    \item \textbf{Sports:} Assessing the probability of winning a game given specific conditions, such as the score at halftime.
  \end{itemize}
\end{frame}

\begin{frame}{Worked Out Problems}
  \textbf{Problem 1:} In a deck of cards, what is the probability of drawing a King, given that the card drawn is a face card?

  \textbf{Solution:}
  \begin{enumerate}
    \item \textbf{Possible Outcomes:} There are 12 face cards in a deck (4 Kings, 4 Queens, 4 Jacks).
    \item \textbf{Favorable Outcomes:} Drawing a King is one favorable outcome.
    \item \textbf{Conditional Probability:} \( P(\text{King}|\text{Face card}) = \frac{\text{Favorable Outcomes}}{\text{Possible Outcomes}} \).
    \item \textbf{Calculate:} \( P(\text{King}|\text{Face card}) = \frac{1}{12} \).
  \end{enumerate}
\end{frame}

\begin{frame}{Reasoning for Conditional Probability Solution}
  \begin{itemize}
    \item Conditional probability is the probability of an event occurring given that another event has already occurred.
    \item For the given problem, the conditional probability of drawing a King, given that the card drawn is a face card, is \(\frac{1}{12}\).
  \end{itemize}
\end{frame}

\begin{frame}{Exercise for Students}
  \begin{enumerate}
    \item In a bag of 20 marbles, 8 are red and 12 are green. If you draw a red marble, what is the probability that the next marble drawn is also red?
    \item Given a pair of six-sided dice, what is the probability of rolling a sum of 7, given that one die shows a 4?
    \item Discuss a real-life scenario where understanding conditional probability is important.
  \end{enumerate}
\end{frame}
