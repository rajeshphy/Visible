\title{Independent Events}
\author{Rajesh Kumar}
\date{}

\begin{frame}
  \titlepage
\end{frame}

\begin{frame}{Independent Events}
  \begin{itemize}
    \item Independent events are events that do not influence each other; the occurrence of one event does not affect the occurrence of the other.
    \item In probability, two events, A and B, are independent if \( P(A \cap B) = P(A) \cdot P(B) \).
  \end{itemize}
\end{frame}

\begin{frame}{Introduction}
  \begin{itemize}
    \item Understanding independence is crucial in probability theory and has applications in various fields.
    \item Independent events simplify probability calculations and are often used in statistical analysis.
  \end{itemize}
\end{frame}

\begin{frame}{Where it is Used in Real Life}
  \begin{itemize}
    \item \textbf{Coin Toss:} Tossing a fair coin multiple times; each toss is independent.
    \item \textbf{Lottery Draws:} Successive draws of lottery numbers are often treated as independent events.
    \item \textbf{Weather Events:} The occurrence of rain today does not influence the probability of rain tomorrow (assuming independence).
  \end{itemize}
\end{frame}

\begin{frame}{Worked Out Problems}
  \textbf{Problem 1:} If you roll a fair six-sided die twice, what is the probability of getting a 4 on the first roll and a 5 on the second roll?

  \textbf{Solution:}
  \begin{enumerate}
    \item \textbf{Probability of Getting 4 on First Roll:} \(P(\text{4 on 1st}) = \frac{1}{6}\).
    \item \textbf{Probability of Getting 5 on Second Roll:} \(P(\text{5 on 2nd}) = \frac{1}{6}\).
    \item \textbf{Multiplication Rule for Independent Events:} \(P(\text{4 on 1st and 5 on 2nd}) = P(\text{4 on 1st}) \cdot P(\text{5 on 2nd})\).
    \item \textbf{Calculate:} \(P(\text{4 on 1st and 5 on 2nd}) = \frac{1}{6} \cdot \frac{1}{6} = \frac{1}{36}\).
  \end{enumerate}
\end{frame}

\begin{frame}{Reasoning for Independent Events Solution}
  \begin{itemize}
    \item Independent events follow the multiplication rule: \(P(A \cap B) = P(A) \cdot P(B)\).
    \item For the given problem, the probability of getting a 4 on the first roll and a 5 on the second roll is calculated using this rule.
    \item \(P(\text{4 on 1st and 5 on 2nd}) = P(\text{4 on 1st}) \cdot P(\text{5 on 2nd})\).
  \end{itemize}
\end{frame}

\begin{frame}{Exercise for Students}
  \begin{enumerate}
    \item If you draw two cards from a standard deck of 52 cards with replacement, what is the probability of getting a spade both times?
    \item In a bag containing 8 red balls and 4 blue balls, what is the probability of drawing two blue balls in succession without replacement?
    \item Discuss a real-life scenario where understanding independent events is important.
  \end{enumerate}
\end{frame}
