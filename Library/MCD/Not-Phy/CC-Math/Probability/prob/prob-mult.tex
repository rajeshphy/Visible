
\title{Multiplication Theorem}
\author{Rajesh Kumar}
\date{}

\begin{frame}
  \titlepage
\end{frame}

\begin{frame}{Multiplication Theorem in Probability}
  \begin{itemize}
    \item The Multiplication Theorem is a fundamental concept in probability that deals with the probability of the intersection of two events.
    \item It is expressed as \(P(A \cap B) = P(A|B) \cdot P(B)\), where \(P(A \cap B)\) is the probability of both events \(A\) and \(B\) occurring, \(P(A|B)\) is the conditional probability of \(A\) given \(B\), and \(P(B)\) is the probability of event \(B\).
  \end{itemize}
\end{frame}

\begin{frame}{Introduction}
  \begin{itemize}
    \item The Multiplication Theorem provides a way to calculate the probability of the joint occurrence of two events.
    \item It is particularly useful when events are dependent on each other.
  \end{itemize}
\end{frame}

\begin{frame}{Where it is Used in Real Life}
  \begin{itemize}
    \item \textbf{Manufacturing:} Probability of producing defective items on a production line.
    \item \textbf{Finance:} Probability of both a stock and its option increasing in value.
    \item \textbf{Epidemiology:} Probability of an individual having two specific health conditions.
  \end{itemize}
\end{frame}

\begin{frame}{Worked Out Problems}
  \textbf{Problem 1:} In a deck of cards, what is the probability of drawing a red card and then drawing a Queen?

  \textbf{Solution:}
  \begin{enumerate}
    \item \textbf{Probability of Drawing a Red Card:} \(P(\text{Red}) = \frac{26}{52} = \frac{1}{2}\).
    \item \textbf{Probability of Drawing a Queen Given a Red Card:} \(P(\text{Queen}|\text{Red}) = \frac{2}{26} = \frac{1}{13}\).
    \item \textbf{Multiplication Theorem:} \(P(\text{Red and Queen}) = P(\text{Queen}|\text{Red}) \cdot P(\text{Red})\).
    \item \textbf{Calculate:} \(P(\text{Red and Queen}) = \frac{1}{13} \cdot \frac{1}{2} = \frac{1}{26}\).
  \end{enumerate}
\end{frame}

\begin{frame}{Reasoning for Multiplication Theorem Solution}
  \begin{itemize}
    \item The Multiplication Theorem states that \(P(A \cap B) = P(A|B) \cdot P(B)\).
    \item For the given problem, the probability of drawing a red card and then drawing a Queen is calculated using the Multiplication Theorem.
    \item \(P(\text{Red and Queen}) = P(\text{Queen}|\text{Red}) \cdot P(\text{Red})\).
  \end{itemize}
\end{frame}

\begin{frame}{Exercise for Students}
  \begin{enumerate}
    \item In a bag containing 5 red balls and 3 blue balls, what is the probability of drawing two red balls in succession without replacement?
    \item If the probability of rain tomorrow is 0.4 and the probability of having a traffic jam on the way to work if it rains is 0.3, what is the probability of both raining and having a traffic jam?
    \item Discuss a real-life scenario where understanding the Multiplication Theorem is important.
  \end{enumerate}
\end{frame}
