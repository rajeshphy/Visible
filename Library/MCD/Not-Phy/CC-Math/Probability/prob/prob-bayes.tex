\title{Bayes' Theorem}
\author{Rajesh Kumar}
\date{}

\begin{frame}
  \titlepage
\end{frame}

\begin{frame}{Bayes' Theorem}
  \begin{itemize}
    \item Bayes' Theorem is a mathematical formula that describes the probability of an event based on prior knowledge of conditions that might be related to the event.
    \item It is named after Thomas Bayes and is a fundamental concept in probability theory and statistics.
  \end{itemize}
\end{frame}

\begin{frame}{Introduction}
  \begin{itemize}
    \item Bayes' Theorem provides a way to update probability estimates based on new evidence.
    \item It is widely used in various fields, including medical diagnosis, information retrieval, and machine learning.
  \end{itemize}
\end{frame}

\begin{frame}{Where it is Used in Real Life}
  \begin{itemize}
    \item \textbf{Medical Diagnosis:} Assessing the probability of a disease given certain symptoms and test results.
    \item \textbf{Email Filtering:} Determining the likelihood of an email being spam based on certain characteristics.
    \item \textbf{Legal System:} Evaluating the probability of guilt or innocence based on evidence presented in a court case.
  \end{itemize}
\end{frame}

\begin{frame}{Bayes' Theorem Formula}
  \begin{block}{Bayes' Theorem}
    \[ P(A|B) = \frac{P(B|A) \cdot P(A)}{P(B)} \]
  \end{block}
  \begin{itemize}
    \item \( P(A|B) \) is the probability of event \( A \) given that event \( B \) has occurred.
    \item \( P(B|A) \) is the probability of event \( B \) given that event \( A \) has occurred.
    \item \( P(A) \) and \( P(B) \) are the probabilities of events \( A \) and \( B \), respectively.
  \end{itemize}
\end{frame}

\begin{frame}{Worked Out Problems}
  \textbf{Problem 1:} A medical test for a certain disease is 95\% accurate. If a person has the disease, the test will correctly identify it 95\% of the time. If a person does not have the disease, the test will incorrectly identify it 5\% of the time. If 2\% of the population has the disease, what is the probability that a person has the disease given that the test result is positive?

  \textbf{Solution:}
  \begin{enumerate}
    \item \textbf{Given:} \( P(Disease) = 0.02 \), \( P(Positive|Disease) = 0.95 \), \( P(Positive|No Disease) = 0.05 \).
    \item \textbf{Calculate:} \( P(Disease|Positive) = \frac{P(Positive|Disease) \cdot P(Disease)}{P(Positive|Disease) \cdot P(Disease) + P(Positive|No Disease) \cdot P(No Disease)} \).
  \end{enumerate}
\end{frame}

\begin{frame}{Reasoning for Bayes' Theorem Solution}
  \begin{itemize}
    \item Bayes' Theorem allows us to update our probability estimate based on new evidence.
    \item For the given medical test problem, the probability of having the disease given a positive test result is calculated using Bayes' Theorem.
  \end{itemize}
\end{frame}

\begin{frame}{Exercise for Students}
  \begin{enumerate}
    \item A factory produces two types of light bulbs: Brand A and Brand B. Brand A bulbs have a defect rate of 3\%, while Brand B bulbs have a defect rate of 5\%. If 80\% of the bulbs produced are Brand A, what is the probability that a randomly selected defective bulb is Brand A?
    \item In a certain city, 2\% of the population has a rare medical condition. A diagnostic test for the condition has a false positive rate of 3\% and a false negative rate of 1\%. If a person tests positive for the condition, what is the probability that they actually have it?
    \item Discuss a real-life scenario where understanding Bayes' Theorem is important.
  \end{enumerate}
\end{frame}
