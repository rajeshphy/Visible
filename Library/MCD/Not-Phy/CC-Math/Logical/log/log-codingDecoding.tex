\title{Coding-Decoding in Logical Reasoning}
\author{Rajesh Kumar}
\date{}

\begin{frame}
  \titlepage
\end{frame}

\begin{frame}{Coding-Decoding in Logical Reasoning}
  \begin{itemize}
    \item Coding-Decoding is a logical reasoning question type where a code or rule is given, and based on that, a message or word needs to be deciphered or coded.
    \item These problems assess a candidate's ability to understand and apply coding rules to decode or encode information.
  \end{itemize}
\end{frame}

\begin{frame}{Introduction}
  \begin{itemize}
    \item Coding-Decoding questions are commonly found in competitive exams and require a logical approach to decipher the given codes.
    \item Understanding the pattern or rule used for coding is crucial for solving these problems.
  \end{itemize}
\end{frame}

\begin{frame}{Types of Coding-Decoding}
  \begin{enumerate}
    \item \textbf{Letter Shifting:} Shifting letters in the alphabet according to a specific pattern.
    \item \textbf{Number Coding:} Assigning numerical values to letters based on a given rule.
    \item \textbf{Substitution Coding:} Substituting letters with other letters or symbols.
    \item \textbf{Mixed Coding:} Combining different coding techniques in a single problem.
  \end{enumerate}
\end{frame}

\begin{frame}{Where it is Used in Real Life}
  \begin{itemize}
    \item \textbf{Cryptography:} Understanding coding and decoding for secure communication.
    \item \textbf{Data Encryption:} Applying coding techniques to protect sensitive information.
    \item \textbf{Programming:} Using coding and decoding in software development for data manipulation.
  \end{itemize}
\end{frame}

\begin{frame}{Strategies for Coding-Decoding}
  \begin{itemize}
    \item \textbf{Analyze the Code:} Examine the given coding rule or pattern carefully.
    \item \textbf{Apply the Rule:} Apply the identified rule to decode or encode the information.
    \item \textbf{Consider Multiple Possibilities:} In mixed coding problems, consider the possibility of multiple coding techniques.
  \end{itemize}
\end{frame}

\begin{frame}{Worked Out Example - Letter Shifting}
  \textbf{Example:} If "CAT" is coded as "FEX," how is "DOG" coded?

  \textbf{Solution:} Each letter in "CAT" is shifted forward by 3 positions, so "DOG" is coded as "GRJ."
\end{frame}

\begin{frame}{Reasoning for Coding-Decoding Solution}
  \begin{itemize}
    \item Coding-Decoding problems test your ability to recognize and apply coding rules.
    \item For the given example, we identified the letter-shifting pattern and applied it logically.
  \end{itemize}
\end{frame}

\begin{frame}{Exercise for Students}
  \begin{enumerate}
    \item Solve a set of letter-shifting coding-decoding problems to practice recognizing patterns.
    \item Explore number coding problems and decipher the numerical values assigned to letters.
    \item Practice substitution coding and mixed coding problems to enhance logical reasoning skills.
  \end{enumerate}
\end{frame}
