\title{Ratio and Proportion}
\author{Rajesh Kumar}
\date{}

\begin{frame}
  \titlepage
\end{frame}

\begin{frame}{Definition}
  \begin{itemize}
    \item \textbf{Ratio:} A comparison of two quantities by division.
    \item \textbf{Proportion:} An equation stating that two ratios are equal.
  \end{itemize}
\end{frame}

\begin{frame}{Introduction}
  \begin{itemize}
    \item Ratios and proportions are fundamental mathematical concepts used for comparing quantities and solving various problems.
    \item They are applicable in a wide range of scenarios, from everyday life to complex mathematical problem-solving.
  \end{itemize}
\end{frame}

\begin{frame}{Where it is Used in Real Life}
  \begin{itemize}
    \item \textbf{Cooking:} Ratios are often used in recipes to determine ingredient quantities.
    \item \textbf{Finance:} Proportions are used in financial calculations, such as interest rates and investment returns.
    \item \textbf{Scale Models:} Architects and model builders use ratios to create accurate scale models.
  \end{itemize}
\end{frame}

\begin{frame}{Worked Out Problems}
  \textbf{Problem 1:} If the ratio of boys to girls in a class is $3:2$, and there are 30 students in total, how many boys and girls are there?

  \textbf{Solution:}
  \begin{enumerate}
    \item \textbf{Given:} Ratio of boys to girls = $3:2$, Total students = $30$.
    \item \textbf{Formula:} Use the ratio to find the individual quantities. Let the number of boys be $3x$ and girls be $2x$.
    \item \textbf{Equation:} $3x + 2x = 30$.
    \item \textbf{Calculate:} Solve for $x$ and find the number of boys and girls.
  \end{enumerate}
\end{frame}

\begin{frame}{Reasoning for Ratio and Proportion Solution}
  \begin{itemize}
    \item The ratio $3:2$ implies that for every $3$ boys, there are $2$ girls.
    \item Assuming the number of boys is $3x$ and girls is $2x$, the total becomes $3x + 2x = 30$.
    \item Solve for $x$ to find the multiplier, and then determine the number of boys and girls accordingly.
  \end{itemize}
\end{frame}

\begin{frame}{Exercise for Students}
  \begin{enumerate}
    \item In a mixture, the ratio of sugar to salt is $5:2$. If there are $35$ units of sugar, how much salt is there in the mixture?
    \item A recipe for a cake requires a ratio of $2:3$ for flour to sugar. If you have $400$ grams of flour, how much sugar is needed?
    \item Discuss a real-life scenario where understanding ratios and proportions is important.
  \end{enumerate}
\end{frame}

