
\title{Simple and Compound Interest}
\author{Rajesh Kumar}
\date{}

\begin{frame}
  \titlepage
\end{frame}


\begin{frame}{Definition}
  \begin{itemize}
    \item \textbf{Simple Interest:} Interest calculated only on the initial principal.
    \item \textbf{Compound Interest:} Interest calculated on both the initial principal and the accumulated interest from previous periods.
  \end{itemize}
\end{frame}

\begin{frame}{Introduction}
  \begin{itemize}
    \item Simple Interest is straightforward and is based only on the original amount of money.
    \item Compound Interest, on the other hand, takes into account the interest that accumulates over time.
  \end{itemize}
\end{frame}

\begin{frame}{Where it is Used in Real Life}
  \begin{itemize}
    \item \textbf{Simple Interest:} Bank savings accounts, loans.
    \item \textbf{Compound Interest:} Investments, credit cards, mortgages.
  \end{itemize}
\end{frame}

\begin{frame}{Worked Out Problems}
  \textbf{Simple Interest:}
  
  If $P$ is the principal amount, $r$ is the rate of interest, and $t$ is the time (in years), then the simple interest ($SI$) is given by the formula:
  \[ SI = P \cdot r \cdot t \]

  \textbf{Compound Interest:}
  
  The compound interest ($CI$) can be calculated using the formula:
  \[ CI = P \left(1 + \frac{r}{n}\right)^{nt} - P \]
\end{frame}

\begin{frame}{Step-by-Step Solution for Simple Interest Problem}
  \textbf{Problem:} Calculate the simple interest for a principal amount of $5000$, an interest rate of $8\%$, and a time period of $3$ years.

  \textbf{Solution:}
  \begin{enumerate}
    \item \textbf{Given:} $P = 5000$, $r = 0.08$, $t = 3$ years.
    \item \textbf{Formula:} Use the simple interest formula $SI = P \cdot r \cdot t$.
    \item \textbf{Substitute:} $SI = 5000 \cdot 0.08 \cdot 3$.
    \item \textbf{Calculate:} $SI = 1200$.
  \end{enumerate}
\end{frame}

\begin{frame}{Reasoning for Simple Interest Solution}
  \begin{itemize}
    \item The formula $SI = P \cdot r \cdot t$ represents the calculation for simple interest.
    \item Substitute the given values into the formula to find the result.
    \item In this case, the principal amount is $5000$, the interest rate is $8\%$, and the time is $3$ years.
    \item Substituting these values gives $SI = 5000 \cdot 0.08 \cdot 3 = 1200$.
    \item Therefore, the simple interest for the given scenario is $1200$.
  \end{itemize}
\end{frame}

\begin{frame}{Exercise for Students}
  \begin{enumerate}
    \item Calculate the simple interest for a principal amount of $5000$, an interest rate of $8\%$, and a time period of $3$ years.
    \item Find the compound interest for a principal amount of $2000$ at an annual interest rate of $6\%$ compounded annually for $4$ years.
    \item Discuss a real-life scenario where understanding simple or compound interest is important.
  \end{enumerate}
\end{frame}

