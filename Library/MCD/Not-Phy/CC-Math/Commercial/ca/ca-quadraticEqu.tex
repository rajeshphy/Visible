\title{Quadratic Equations}
\author{Rajesh Kumar}
\date{}

\begin{frame}
  \titlepage
\end{frame}

\begin{frame}{Definition}
  \begin{itemize}
    \item \textbf{Quadratic Equation:} A second-degree polynomial equation in a single variable with the form $ax^2 + bx + c = 0$, where $a$, $b$, and $c$ are constants.
  \end{itemize}
\end{frame}

\begin{frame}{Introduction}
  \begin{itemize}
    \item Quadratic equations are an essential part of algebra and are widely used to model various real-world phenomena.
    \item They describe the shape of a parabola and have applications in physics, engineering, and other fields.
  \end{itemize}
\end{frame}

\begin{frame}{Where it is Used in Real Life}
  \begin{itemize}
    \item \textbf{Physics:} Projectile motion can be modeled using quadratic equations.
    \item \textbf{Engineering:} Quadratic equations describe the behavior of certain mechanical systems.
    \item \textbf{Finance:} Quadratic equations are used in financial modeling.
  \end{itemize}
\end{frame}

\begin{frame}{Worked Out Problems}
  \textbf{Problem 1:} Solve the quadratic equation $x^2 - 4x + 4 = 0$.

  \textbf{Solution:}
  \begin{enumerate}
    \item \textbf{Equation:} $x^2 - 4x + 4 = 0$.
    \item \textbf{Factor:} The equation can be factored into $(x - 2)^2 = 0$.
    \item \textbf{Solve for $x$:} Set $x - 2 = 0$, leading to $x = 2$.
  \end{enumerate}
\end{frame}

\begin{frame}{Reasoning for Quadratic Equation Solution}
  \begin{itemize}
    \item To solve a quadratic equation, factor it and set each factor equal to zero.
    \item For the given problem, the quadratic equation $x^2 - 4x + 4 = 0$ can be factored into $(x - 2)^2 = 0$.
    \item Set $x - 2 = 0$ to find the solution $x = 2$.
  \end{itemize}
\end{frame}

\begin{frame}{Exercise for Students}
  \begin{enumerate}
    \item Solve the quadratic equation $2x^2 - 5x + 2 = 0$.
    \item Write a quadratic equation given the roots $x = 3$ and $x = -2$.
    \item Discuss a real-life scenario where understanding quadratic equations is important.
  \end{enumerate}
\end{frame}

\end{document}
