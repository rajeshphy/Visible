\title{Computer Networks}
\author{Rajesh Kumar}
\date{}

\begin{frame}
  \titlepage
\end{frame}

\begin{frame}{Introduction}
  \begin{itemize}
    \item A computer network is a collection of interconnected devices that can communicate and share resources.
    \item Overview of the importance and types of computer networks.
  \end{itemize}
\end{frame}

\begin{frame}{Types of Computer Networks}
  \begin{enumerate}
    \item \textbf{Local Area Network (LAN):}
      \begin{itemize}
        \item Limited geographical area (e.g., within a building).
        \item High data transfer rates, commonly used in offices and homes.
      \end{itemize}
    \item \textbf{Wide Area Network (WAN):}
      \begin{itemize}
        \item Spans a large geographical area (e.g., across cities or countries).
        \item Relies on public or private network infrastructure.
      \end{itemize}
    \item \textbf{Wireless Networks:}
      \begin{itemize}
        \item Connect devices without physical cables.
        \item Wi-Fi, Bluetooth, and cellular networks fall into this category.
      \end{itemize}
    \item \textbf{Metropolitan Area Network (MAN):}
      \begin{itemize}
        \item Covers a larger geographical area than a LAN but smaller than a WAN (e.g., a city).
      \end{itemize}
  \end{enumerate}
\end{frame}

\begin{frame}{Components of a Computer Network}
  \begin{enumerate}
    \item \textbf{Nodes:} Devices connected to the network (e.g., computers, routers, servers).
    \item \textbf{Links:} Communication channels that connect nodes (e.g., wired or wireless connections).
    \item \textbf{Switches and Routers:} Devices that facilitate data transfer within the network.
    \item \textbf{Protocols:} Rules and conventions governing communication between devices.
    \item \textbf{Topologies:} Physical or logical layout of the network (e.g., star, bus, ring).
  \end{enumerate}
\end{frame}

\begin{frame}{Network Security}
  \begin{itemize}
    \item \textbf{Importance of Security:} Protecting data, preventing unauthorized access, and ensuring the integrity of the network.
    \item \textbf{Firewalls and Encryption:} Implementing measures to safeguard against cyber threats.
    \item \textbf{Access Control:} Managing user permissions and restricting unauthorized access.
    \item \textbf{Intrusion Detection Systems (IDS):} Monitoring and detecting suspicious activities within the network.
  \end{itemize}
\end{frame}

\begin{frame}{Challenges and Future Trends}
  \begin{itemize}
    \item \textbf{Scalability:} Handling the increasing number of connected devices.
    \item \textbf{Reliability:} Ensuring continuous and reliable network connectivity.
    \item \textbf{5G Technology:} The next generation of wireless communication.
    \item \textbf{Internet of Things (IoT):} Connecting everyday devices to the network.
    \item \textbf{Software-Defined Networking (SDN):} Centralized network management for flexibility and efficiency.
  \end{itemize}
\end{frame}

\begin{frame}{Conclusion}
  \begin{itemize}
    \item Computer networks are the backbone of modern communication and information exchange.
    \item Understanding their types, components, and security measures is crucial for effective network management.
    \item Ongoing advancements will shape the future of computer networks, influencing connectivity on a global scale.
  \end{itemize}
\end{frame}
