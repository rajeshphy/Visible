\documentclass{beamer}
\usetheme{CambridgeUS}

\title{Arithmetic, Probability, Statistics, Logical Reasoning}
\author{Rajesh Kumar\\kr.rajesh.phy@gmail.com\\Model College Dumka}
\date{\today}

\begin{document}

\frame{\titlepage}

\begin{frame}
    \frametitle{Arithmetic}
    % Definition of Arithmetic
    {\bf Definition of Arithmetic:} Arithmetic is a branch of mathematics that consists of the study of numbers, especially the properties of the traditional operations on them---addition, subtraction, multiplication and division. Arithmetic is an elementary part of number theory, and number theory is considered to be one of the top-level divisions of modern mathematics, along with algebra, geometry, and analysis. The terms arithmetic and higher arithmetic were used until the beginning of the 20th century as synonyms for number theory and are sometimes still used to refer to a wider part of number theory.\\ 
\end{frame}

\begin{frame}
    \frametitle{Arithmetic-Percentage-I}
    % Definition of Percentage
    {\bf Definition of Percentage:} In mathematics, a percentage is a number or ratio expressed as a fraction of 100. It is often denoted using the percent sign, "\%", or the abbreviation "pct.". For example, 45\% (read as "forty-five percent") is equal to 45/100, or 0.45.\\
    % Simple day today examples of Percentage
    {\bf Simple day today examples of Percentage:} A percent is a ratio whose second term is 100. Percent means parts per hundred. The word comes from the Latin phrase per centum, which means per hundred.\\
\end{frame}

\begin{frame}
    \frametitle{Arithmetic-Percentage-II}
    % Percentage to Fraction
    {\bf Percentage to Fraction:} To convert a percentage to a fraction, first convert to a decimal (divide by 100), then use the steps for converting decimal to fractions (like above). Example: 25\% = 25/100 = 1/4.\\
    % Percentage to Decimal
    {\bf Percentage to Decimal:} To convert from percent to decimal: Divide the percentage number by 100 to get a decimal number. Move the decimal point to the right 2 places.\\
    % Percentage to Ratio
    {\bf Percentage to Ratio:} To convert from percent to ratio: Write the percentage as a fraction out of 100. Simplify the fraction if possible. Write it as a ratio using the ":" symbol.\\
\end{frame}

%Percentage numerical problems with solutions
\begin{frame}
    \frametitle{Arithmetic-Percentage-III}
    % Percentage numerical problems with solutions
    {\bf Percentage numerical problems with solutions:} 
    \begin{enumerate}
        \item 10\% of 100 = 10
        \item 10\% of 200 = 20
        \item 10\% of 300 = 30
        \item 10\% of 400 = 40
        \item 10\% of 500 = 50
        \item 10\% of 600 = 60
        \item 10\% of 700 = 70
        \item 10\% of 800 = 80
        \item 10\% of 900 = 90
        \item 10\% of 1000 = 100
    \end{enumerate}
\end{frame}

%Percentage numerical problems assignment
\begin{frame}
    \frametitle{Arithmetic-Percentage-IV}
    % Percentage numerical problems assignment
    {\bf Percentage numerical problems assignment:} 
    \begin{enumerate}
        \item 10\% of 100 = ?
        \item 10\% of 200 = ?
        \item 10\% of 300 = ?
        \item 10\% of 400 = ?
        \item 10\% of 500 = ?
        \item 10\% of 600 = ?
        \item 10\% of 700 = ?
        \item 10\% of 800 = ?
        \item 10\% of 900 = ?
        \item 10\% of 1000 = ?
    \end{enumerate}
\end{frame}

% Part-II

\begin{frame}
    \frametitle{Probability}
    %Definition of Probability
    {\bf Definition of Probability:} Probability is the measure of the likelihood that an event will occur. Probability is quantified as a number between 0 and 1, where, loosely speaking, 0 indicates impossibility and 1 indicates certainty. The higher the probability of an event, the more likely it is that the event will occur.\\
\end{frame}
%   Different types of Probability-Conditional probability, Mutually exclusive events, Independent events, Dependent events, Addition law of probability, Multiplication law of probability, Bayes' theorem
\begin{frame}
    \frametitle{Probability-Conditional probability}
    %Definition of Conditional probability
    {\bf Definition of Conditional probability:} In probability theory, conditional probability is a measure of the probability of an event occurring given that another event has (by assumption, presumption, assertion or evidence) occurred. If the event of interest is A and the event B is known or assumed to have occurred, "the conditional probability of A given B", or "the probability of A under the condition B", is usually written as P(A|B), or sometimes PB(A) or P(A/B).\\
\end{frame}

\begin{frame}
    \frametitle{Probability-Mutually exclusive events}
    %Definition of Mutually exclusive events
    {\bf Definition of Mutually exclusive events:} In probability theory, two events are mutually exclusive or disjoint if they cannot both occur at the same time. A clear example is the set of outcomes of a single coin toss, which can result in either heads or tails, but not both.\\
\end{frame}

\begin{frame}
    \frametitle{Probability-Independent events}
    %Definition of Independent events
    {\bf Definition of Independent events:} In probability theory, to events are independent if the occurrence of one does not affect the probability of occurrence of the other. Similarly, two random variables are independent if the realization of one does not affect the probability distribution of the other.\\
\end{frame}

\begin{frame}
    \frametitle{Probability-Dependent events}
    %Definition of Dependent events
    {\bf Definition of Dependent events:} In probability theory, two events are dependent if the probability of one event happening influences the probability of the other event.\\
\end{frame}


\begin{frame}
    \frametitle{Probability-Bayes' theorem}
    %Definition of Bayes' theorem
    {\bf Definition of Bayes' theorem:} In probability theory and statistics, Bayes' theorem (alternatively Bayes' law or Bayes' rule) describes the probability of an event, based on prior knowledge of conditions that might be related to the event. For example, if cancer is related to age, then, using Bayes' theorem, a person's age can be used to more accurately assess the probability that they have cancer, compared to the assessment of the probability of cancer made without knowledge of the person's age.\\
\end{frame}

%Bayes' theorem numerical problems with solutions

\begin{frame}
    \frametitle{Probability-Bayes' theorem-I}
    %Bayes' theorem numerical problems with solutions
    {\bf Bayes' theorem numerical problems with solutions:} 
    Suppose a test for using a particular drug is 99\% sensitive and 99\% specific. That is, the test will produce 99\% true positive results for drug users and 99\% true negative results for non-drug users. Suppose that 0.5\% of people are users of the drug. What is the probability that a randomly selected individual with a positive test is a drug user?\\

    {\bf Solution:} Let D be the event that the person is a drug user and T be the event that the person tests positive. We need to compute $P(D|T)$. By Bayes' theorem, we have
    \begin{equation}
        P(D|T) = \frac{P(T|D)P(D)}{P(T)}
    \end{equation}
\end{frame}

\begin{frame}
    \frametitle{Statistics}
    % Include your content related to Statistics here
\end{frame}

\begin{frame}
    \frametitle{Logical Reasoning}
    % Include your content related to Logical Reasoning here
\end{frame}

\end{document}
