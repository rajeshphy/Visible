\documentclass[a4paper, 12pt]{article}
\usepackage[margin=2cm]{geometry}
\usepackage{amssymb, amsmath, graphicx, tabularx, booktabs, subfiles, amsthm,enumitem,xcolor,appendix}
 

\newcommand{\ans}[1]{\textcolor{red}{#1}}

\newcommand{\boxx}[1]{%
    \begin{tcolorbox}[colback=gray!10, colframe=gray!50, title=Where it is getting wrong?]
        #1
    \end{tcolorbox}%
}


\title{Electrostatics}
\author{}

\begin{document}
\maketitle
%------------Body-Starts--------------------
\section*{Laplace Equation in Rectangular, Spherical, and Cylindrical Coordinates}
The Laplace equation is a partial differential equation that describes the behavior of scalar fields. In three-dimensional space, it is given by:

1. \textbf{Rectangular Coordinates (Cartesian Coordinates):}
   \[ \nabla^2 \phi = \frac{\partial^2 \phi}{\partial x^2} + \frac{\partial^2 \phi}{\partial y^2} + \frac{\partial^2 \phi}{\partial z^2} = 0 \]

2. \textbf{Spherical Coordinates:}
   \[ \nabla^2 \phi = \frac{1}{r^2} \frac{\partial}{\partial r} \left(r^2 \frac{\partial \phi}{\partial r}\right) + \frac{1}{r^2 \sin \theta} \frac{\partial}{\partial \theta} \left(\sin \theta \frac{\partial \phi}{\partial \theta}\right) + \frac{1}{r^2 \sin^2 \theta} \frac{\partial^2 \phi}{\partial \phi^2} = 0 \]

3. \textbf{Cylindrical Coordinates:}
   \[ \nabla^2 \phi = \frac{1}{r} \frac{\partial}{\partial r} \left(r \frac{\partial \phi}{\partial r}\right) + \frac{1}{r^2} \frac{\partial^2 \phi}{\partial \theta^2} + \frac{\partial^2 \phi}{\partial z^2} = 0 \]

In these equations:
- \(\phi\) is the scalar field,
- \(\nabla^2\) is the Laplacian operator,
- \(r\) is the radial distance in spherical and cylindrical coordinates,
- \(\theta\) is the polar angle in spherical coordinates,
- \(x\), \(y\), and \(z\) are the Cartesian coordinates.


\section*{Explain electric dipole, quadrupole and their moments}

1. {\bf Electric Dipole:}
   - An electric dipole consists of two equal and opposite point charges \(q\) and \(-q\) separated by a distance \(d\).
   - The electric dipole moment (\(p\)) is a vector pointing from the negative charge to the positive charge and is given by:
     \[ \mathbf{p} = q \cdot \mathbf{d} \]
     where \(\mathbf{p}\) is the electric dipole moment vector, \(q\) is the magnitude of each charge, and \(\mathbf{d}\) is the vector pointing from the negative to the positive charge.

2. {\bf Electric Quadrupole:}
   - An electric quadrupole involves a system of two dipoles or four point charges arranged in a specific way.
   - The electric quadrupole moment (\(Q\)) is a tensor quantity and is represented by a matrix. For a system of point charges \(q_1, q_2, q_3, q_4\) located at \(\mathbf{r}_1, \mathbf{r}_2, \mathbf{r}_3, \mathbf{r}_4\), the quadrupole moment tensor elements are given by:
     \[ Q_{ij} = \sum_{k=1}^4 q_k (3r_{ki} r_{kj} - \delta_{ij} r_k^2) \]
     where \(r_{ki} = |\mathbf{r}_k - \mathbf{r}_i|\) is the distance between the \(i\)-th charge and the \(k\)-th charge, and \(\delta_{ij}\) is the Kronecker delta.

These moments are useful in describing the electric field produced by a distribution of charges. For example, the electric field (\(\mathbf{E}\)) produced by an electric dipole at a point in space is given by:
\[ \mathbf{E} = \frac{1}{4\pi \varepsilon_0} \left( \frac{3(\mathbf{p} \cdot \mathbf{r}) \mathbf{r} - \mathbf{p}r^2}{r^5} \right) \]
where \(\varepsilon_0\) is the permittivity of free space, \(\mathbf{p}\) is the dipole moment, and \(\mathbf{r}\) is the position vector from the dipole to the point in space.

\section*{Obtain an expression for potential and intensity of dipole}

The electric potential (\(V\)) and electric field intensity (\(E\)) produced by an electric dipole can be derived using the concept of a point charge. Let's consider an electric dipole with a dipole moment \(\mathbf{p}\), which consists of two charges \(+q\) and \(-q\) separated by a distance \(2a\).

1. {\bf Electric Potential (\(V\)) at a Point P:}
   - The electric potential at a point \(P\) due to a point charge \(q\) at a distance \(r\) is given by Coulomb's law:
     \[ V_q = \frac{1}{4\pi\varepsilon_0} \frac{q}{r} \]
   - The potential \(V\) at point \(P\) due to the dipole is the sum of the potentials produced by the positive and negative charges:
     \[ V = V_{+q} + V_{-q} \]
   - Assuming that \(P\) is at a distance \(r\) much greater than the separation \(2a\) (\(r \gg a\)), we can use the binomial expansion to approximate \(1/r\) terms:
     \[ V \approx \frac{1}{4\pi\varepsilon_0} \left( \frac{q}{r - a} - \frac{q}{r + a} \right) \]
   - Simplifying further, we get the expression for the electric potential due to a dipole at large distances:
     \[ V \approx \frac{1}{4\pi\varepsilon_0} \frac{\mathbf{p} \cdot \mathbf{r}}{r^3} \]
     where \(\mathbf{p}\) is the dipole moment vector and \(\mathbf{r}\) is the position vector pointing from the dipole to the point \(P\).

2. {\bf Electric Field Intensity (\(E\)) at a Point P:}
   - The electric field intensity at a point \(P\) is the negative gradient of the electric potential:
     \[ \mathbf{E} = -\nabla V \]
   - For a dipole at a large distance (\(r \gg a\)), the expression for the electric field intensity is:
     \[ \mathbf{E} \approx \frac{1}{4\pi\varepsilon_0} \frac{3(\mathbf{p} \cdot \mathbf{r})\mathbf{r} - \mathbf{p}r^2}{r^5} \]

These expressions are valid when the observation point \(P\) is far away from the dipole compared to the separation distance \(2a\), ensuring that the dipole can be approximated as a point dipole at the center of the dipole axis.


\section*{Expression for potential and electric field of quadrupole}

{\bf 1. Quadrupole Moment:}

Consider a system of four point charges \(q_1, q_2, q_3, q_4\) located at positions \(\mathbf{r}_1, \mathbf{r}_2, \mathbf{r}_3, \mathbf{r}_4\). The quadrupole moment \(Q_{ij}\) for this system is given by:

\[ Q_{ij} = \sum_{k=1}^4 q_k \left(3r_{ki} r_{kj} - \delta_{ij} r_k^2\right) \]

Here,
- \(r_{ki} = |\mathbf{r}_k - \mathbf{r}_i|\) is the distance between the \(i\)-th charge and the \(k\)-th charge.
- \(\delta_{ij}\) is the Kronecker delta.

{\bf 2. Electric Potential (\(V\)) due to a Quadrupole:}

The electric potential at a point \(P\) located at position vector \(\mathbf{r}\) due to a quadrupole is given by:

\[ V = \frac{1}{4\pi\varepsilon_0} \sum_{i,j=1}^3 \frac{Q_{ij} r_i r_j}{r^5} \]

Now, let's expand this expression:

\[ V = \frac{1}{4\pi\varepsilon_0} \sum_{i,j=1}^3 \frac{1}{r^5} \left(\sum_{k=1}^4 q_k \left(3r_{ki} r_{kj} - \delta_{ij} r_k^2\right)\right) r_i r_j \]

\[ V = \frac{1}{4\pi\varepsilon_0 r^5} \sum_{i,j=1}^3 \left(\sum_{k=1}^4 q_k \left(3r_{ki} r_{kj} - \delta_{ij} r_k^2\right)\right) r_i r_j \]

{\bf 3. Electric Field Intensity (\(\mathbf{E}\)) due to a Quadrupole:}

The electric field intensity \(\mathbf{E}\) is obtained by taking the negative gradient of the electric potential:

\[ \mathbf{E} = -\nabla V \]

The \(i\)-th component of the electric field, \(E_i\), is given by:

\[ E_i = -\frac{\partial V}{\partial r_i} \]

Now, let's differentiate the potential with respect to \(r_i\):

\[ \frac{\partial V}{\partial r_i} = \frac{\partial}{\partial r_i} \left(\frac{1}{4\pi\varepsilon_0 r^5} \sum_{i,j=1}^3 \left(\sum_{k=1}^4 q_k \left(3r_{ki} r_{kj} - \delta_{ij} r_k^2\right)\right) r_i r_j\right) \]

This involves a series of partial derivatives and the product rule, leading to a complex expression.

\[ E_i = -\frac{\partial V}{\partial r_i} \]

{\bf Summary}

The expressions for the electric potential and electric field due to a quadrupole involve the quadrupole moment and are derived by considering the contributions from each point charge in the system. The expressions are complex and involve multiple summations, making numerical or analytical simplifications necessary for specific configurations. The key idea is to understand how each charge contributes to the potential and, subsequently, to the electric field.
%------------Body-Ends--------------------
%\bibliography{ref.bib}
%\bibliographystyle{IEEEtran}
\end{document}
