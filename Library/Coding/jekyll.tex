\documentclass{book}

\usepackage{geometry}
\usepackage{hyperref}
\usepackage{listings}
\usepackage{color}
\usepackage{xcolor}
\usepackage{graphicx}

\geometry{a4paper, margin=1in}

\title{Building Websites with Jekyll}
\author{Rajesh Kumar\\kr.rajesh.phy@gmail.com}
\date{\today}

\begin{document}

\maketitle

\section*{Introduction}

Jekyll is a powerful static site generator that simplifies the process of building and maintaining websites. It's a popular choice for developers who want a fast and efficient way to create static websites or blogs without the complexity of a traditional content management system.

In this article, we'll explore the basics of Jekyll and how you can use it to develop your website.

\section*{Getting Started with Jekyll}

\subsection*{Installation}

Before you start using Jekyll, you need to have Ruby installed on your system. Once Ruby is installed, you can install Jekyll using the following command:

\begin{lstlisting}[language=bash]
gem install jekyll
\end{lstlisting}

\subsection*{Creating a New Jekyll Site}

To create a new Jekyll site, navigate to the desired directory and run the following command:

\begin{lstlisting}[language=bash]
jekyll new mywebsite
\end{lstlisting}

This will generate a new Jekyll site in the "mywebsite" directory.

\section*{Understanding the Jekyll Directory Structure}

Jekyll follows a specific directory structure. Here are some key directories and files:

\begin{itemize}
    \item \textbf{\_layouts}: Contains templates for different pages.
    \item \textbf{\_posts}: Stores your blog posts in Markdown format.
    \item \textbf{\_config.yml}: Configuration file for your Jekyll site.
    \item \textbf{index.html}: The main page of your site.
\end{itemize}

\section*{Creating Content}

To add a new blog post, create a new Markdown file in the \_posts directory. Jekyll will automatically convert these files into HTML.

\section*{Customizing Your Site}

Jekyll allows you to customize your site's appearance and functionality easily. You can modify the \_config.yml file to change settings such as the site title, description, and more.

\section*{Building and Previewing Your Site}

To build and preview your site locally, run the following commands:

\begin{lstlisting}[language=bash]
cd mywebsite
jekyll serve
\end{lstlisting}

This will start a local server, and you can view your site by navigating to \url{http://localhost:4000} in your web browser.

\section*{Conclusion}

Jekyll is a fantastic tool for website development, offering simplicity and flexibility. Whether you're creating a personal blog or a portfolio site, Jekyll can streamline the process and help you focus on creating great content.

In future posts, we'll explore advanced features of Jekyll and how to enhance your site further.

Happy coding!

\chapter*{Practical Session}


% How to install Jekyll
\section*{How to install Jekyll in Linux}
% Mention the steps to install Jekyll in terminal in Ubuntu
\begin{enumerate}
\item \begin{verbatim}
sudo apt-get update
\end{verbatim}
\item \begin{verbatim}
sudo apt-get install ruby-full build-essential zlib1g-dev
\end{verbatim}
\item \begin{verbatim}
echo '# Install Ruby Gems to ~/gems' >> ~/.bashrc
\end{verbatim}
\item \begin{verbatim}
echo 'export GEM_HOME="~/gems"' >> ~/.bashrc
\end{verbatim}
\item \begin{verbatim}
echo 'export PATH="~/gems/bin:$PATH"' >> ~/.bashrc
\end{verbatim}
\item \begin{verbatim}
source ~/.bashrc
\end{verbatim}
\item \begin{verbatim}
gem install jekyll bundler
\end{verbatim}
\item \begin{verbatim}
jekyll new my-awesome-site
\end{verbatim}
\end{enumerate}

%For Mac OS
\section*{How to install Jekyll on macOS}
\begin{enumerate}
\item First, check if gem is installed or not. Type the following command to check if gem is installed or not:
 \begin{verbatim} 
sudo gem install jekyll 
\end{verbatim}
\item If gem is not installed, then install it using this command:
 \begin{verbatim} 
sudo gem install jekyll 
\end{verbatim}

\item Ensure the gem directory is in the path. Add the following line to your shell profile file (e.g., ~/.bashrc or ~/.zshrc):
 \begin{verbatim} 
export PATH="$PATH:$(ruby -e 'puts Gem.user_dir')/bin"
\end{verbatim}

\item Install Jekyll and Bundler using the following command:
 \begin{verbatim}gem install bundler
\end{verbatim}
\item Install Jekyll Dependences using the following command:
 \begin{verbatim}bundle install
\end{verbatim}
\item Create a new Jekyll site at ./myblog:
 \begin{verbatim}jekyll new myblog
\end{verbatim}

\end{enumerate}

%For Windows
\section*{How to install Jekyll on Windows}
\begin{enumerate}
\item Install Ruby+Devkit 2.6.X (x64) from RubyInstaller Downloads. Choose the default options for installation.
\item Run the ridk install step from the last stage of the RubyInstaller installation. This is needed for installing gems with native extensions. From the command prompt:
 \begin{verbatim}ridk install
\end{verbatim}  
\item Install Jekyll and Bundler using the following command:
 \begin{verbatim}gem install jekyll bundler
\end{verbatim}
\item Create a new Jekyll site at ./myblog:
 \begin{verbatim}jekyll new myblog
\end{verbatim}
\item Change into your new directory:
 \begin{verbatim}cd myblog
\end{verbatim}
\item Build the site and make it available on a local server:
 \begin{verbatim}bundle exec jekyll serve
\end{verbatim}
\item Browse to \url{http://localhost:4000}
\item Add the following line to your shell profile file (e.g., ~/.bashrc or ~/.zshrc):
 \begin{verbatim}export PATH="$PATH:$(ruby -e 'puts Gem.user_dir')/bin"
\end{verbatim}
\end{enumerate}

\end{document}
